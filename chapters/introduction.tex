% Contextualizar: do geral oara o específico, do que se trata o teu trabalho - 5 a 10 parágrafos contextualizando
% Qual é o problema, por que o problema é um problema
% Objetivos Geral e específicos
% Motivação
% Justificativa
% Metodologia de pesquisa adotada
% Opcional: listar as principais contribuicoes
% Opcional: Limitaçoes do teu trabalho, no sentido do que nao é o teu trabalho
% Apresentacao o conteudo da dissertacao

% Motivação: por que trabalhar nesse tema. "trabalho de cooperação itnernacional labsec com a universidade de carleton, onde foram feitos vários trabalhos na área de processamento rápido de corpos finitos". aspeco social, pessoal, laboratório, pesquisa. "durante os anos as pesquisam sendo realizadas no grupo de pesquisas do labsec, chegou-se a conclusao que o desempenho e performance de algoritmos não era o desejado, com a vinda e internacionalização do grupo de pesquisa especial, e cooperação com a unviersidade de Carleton, descobriu-se que haviam alguams coisas a fazer..."
% Justificativa: científiac. É importante melhorar o desempenho pra trabalhar na aritmética em corpos finitos? Algoritmos são as peças chaves, diminuir úmero de operacóes leva a melhor performance...

% Näo incluir áreas de pesquisa que não deram resultado: shamir, ECC, etc. Se tiver alguma coisa na mesma área, incluir nos trabalhos futuros