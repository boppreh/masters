% Discutir as contribuições. O que foi trazido de melhorias? Avaliação do trabalho. Contextualizar a contribuição, com análise crítical.

% Objetivos atingidos: específicos e gerais.
% Propostas futuras.

% Relacionar os objetivos específicos ( cada um deles ) com o que foi feito e onde está isso no trabalho, de forma critica

This work had a twofold goal of being an introduction to finite field arithmetic, and proposing efficient algorithms for this area. \\

We hope to have achieved the first goal by giving a humble but self-contained introduction to the mathematical structures involved. Furthermore, during the presentation of our contributions we tried to keep the development and reasoning behind it as clear as possible. Finally, the visualization tool helps giving intuition to the operations involved, and acting as an exploratory tool. \\

\section{Conclusion}

Our second goal was to improve finite field-based cryptosystems by creating more efficient algorithms. In this area we made three main contributions: an efficient algorithm for reduction modulo any irreducible binary polynomial; a squaring algorithm that requires less operations; and a set of helper functions to translate bit-level algorithms to software implementations. \\

Our reduction algorithm is not more efficient than the state of the art, but is more general. This gives more flexibility on the choice of irreducible polynomial, which can be used to optimize other properties of the field structure, or to avoid weak structures. \\

In this work we present a low-complexity bit-parallel squarer algorithm. It is able to match the state of the art~\cite{wu2002bit} in XOR gate count when $GF(2^m)$ is defined using trinomials. Xiong~\cite{xiong2014gf} achieves lower XOR gate complexity when pentanomials are used; however it uses a special polynomial basis and is suitable to only a small fraction of all irreducible pentanomials, while ours is general for any low-weight polynomial.\\

Park~\cite{park2012explicit} also proposes a general squaring algorithm focusing on pentanomials with $a \leq \ceil{m/2}$. Her squarer is the best available in the literature considering both the number of XOR operations and the delay. However, our algorithm is able to match the circuit delay on some polynomials and has a strictly smaller number of XORs than the upper bound provided, while being suitable not only for pentanomials. 
We remark that our squarer works for any general low-weight $k$-nomial. As mentioned in Section~\ref{squaring} there is not much work in the literature for this type of $k$-nomial squarer, $k>5$, but our algorithm can still be efficiently used in these cases.\\

One disadvantage of our proposed squarer is that it may produce higher delays, up to $6 T_X$ for pentanomials. This can be mitigated in two ways. First, by narrowing the selection of irreducible pentanomial. Indeed,  pentanomials that result in a delay of $3 T_X$ are abundant. Second, the resulting circuit can be modified to reduce the delay in exchange for some extra XORs. An automated way to perform this tweak is still an open problem, as is the exact formula for circuit delays.\\

We also researched analogous word-level algorithms. Unfortunately we were not able to create algorithms with better efficiency than the state of the art. Still, our research yielded useful knowledge and tools for automatically adapting bit-level algorithms. \\

Finally, our visualization tool was a surprisingly useful to get a better understanding of the problem. It enabled us to move faster during the exploratory phase, and removed classes of errors by automating analysis processes. \\


\section{Future work}

Many venues of research were not explored adequately in this work, mainly due to time restrictions. First, our squarer algorithm may be adapted to perform more efficient exponentiations, a technique that has been demonstrated before~\ref{artigo_park_exp}. The contributions on $p$-th power may also be used in similar fashion, but the path is not as clear. \\

Similarly, our reduction algorithm can be adapted to some multiplication algorithms. This could lead to further optimizations by finding common expressions between the reduction and the rest of the multiplication. \\

Our research into word-level algorithms could be greatly improved by the addition of SIMD (Single Instruction Multiple Data) instructions. Although not available in every platform, such instructions can drastically enhance the performance of algorithms like this. It would also be interesting to see the helper functions applied to other kinds of algorithms. \\

The visualization tool could be adapted to support different types of algorithms and circuits, such as full multiplication algorithms, or finite fields with different characteristics. The circuit visualization version is specific to our proposed algorithm, but could be used for different circuits. \\

Finally, the mathematical introduction to finite fields covers only the minimum necessary to give the background understanding for our research. It could stand to have improvements and additions, and hopefully attract more researchers from our program to this field of study.