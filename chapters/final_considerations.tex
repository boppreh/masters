% Relacionar os objetivos específicos ( cada um deles ) com o que foi feito e onde está isso no trabalho, de forma critica
% Trabalhos futuros: pelos menos apresentar 5 trabalhos futuros

\section{Conclusion} \label{conclusion}

In this work we present a low-complexity bit-parallel squarer algorithm. It is able to match the state of the art~\cite{wu2002bit} in XOR gate count when $GF(2^m)$ is defined using trinomials. Xiong~\cite{xiong2014gf} achieves lower XOR gate complexity when pentanomials are used; however it uses a special polynomial basis and is suitable to only a small fraction of all irreducible pentanomials, while ours is general for any low-weight polynomial.

Park~\cite{park2012explicit} also proposes a general squaring algorithm focusing on pentanomials with $a \leq \ceil{m/2}$. Her squarer is the best available in the literature considering both the number of XOR operations and the delay. However, our algorithm is able to match the circuit delay on some polynomials and has a strictly smaller number of XORs than the upper bound provided, while being suitable not only for pentanomials. 
We remark that our squarer works for any general low-weight $k$-nomial. As mentioned in Section~\ref{squaring} there is not much work in the literature for this type of $k$-nomial squarer, $k>5$, but our algorithm can still be efficiently used in these cases.

One disadvantage of our proposed squarer is that it may produce higher delays, up to $6 T_X$ for pentanomials. This can be mitigated in two ways. First, by narrowing the selection of irreducible pentanomial. Indeed,  pentanomials that result in a delay of $3 T_X$ are abundant. Second, the resulting circuit can be modified to reduce the delay in exchange for some extra XORs. An automated way to perform this tweak is still an open problem, as is the exact formula for circuit delays.

In conclusion, our algorithms give more freedom to select irreducible polynomials while maintaining competitive efficiency.
