\subsection{Finite fields}

A Finite Field $\F_{p^m}$ (also known as Galois Field $GF(p^m)$) is an abstract mathematical structure composed of a set of elements and two operators:

\[
\F_{p^m} = (\{A_0, A_1, A_2, ... A_{p^m-1}\}, +, \cdot)
\]

\begin{itemize}
\item $\F_{p^m}$: A finite field of size $p^m$, where:
    \begin{itemize}
    \item $p$: the \emph{field characteristic}, $p \in \mathbb{P}$ (i.e. must be a prime number);
    \item $m$: the \emph{extension size} $m \geq 1, m \in \mathbb{N}$;
    \end{itemize}
\item $\{A_0, A_1, A_2, ... A_{p^m-1}\}$: A set of $p^m$ \emph{field elements}. In this work we use the names $A$, $B$ and $C$ to refer to arbitrary members of this set, and $\F_{p^m}$ as an informal shorthand for this set (e.g. $A \in \F_{p^m}$).
\item $+, \cdot$~: Addition and multiplication operators, where:
    \begin{itemize}
    \item they are binary operators that operate field elements, $+, \cdot : (\F_{p^m} \times \F_{p^m}) \mapsto \F_{p^m}$;
    \item there are two elements labeled $0$ and $1$ that behave as the additive and multiplicative identities, $0 \neq 1$, such that $\forall A (A + 0 = A~\wedge~A \cdot 1 = A)$;
    \item both operations commute (i.e. $A + B = B + A$ and $A \cdot B = B \cdot A$) and associate (i.e. $(A + B) + C = A + (B + C)$ and $(A \cdot B) \cdot C = A \cdot (B \cdot C)$);
    \item every element has an additive and a multiplicative inverse, also in the field (i.e. there are elements $-A$ and $A^{-1}$ such that $A + -A = 0$ and $A \cdot A^{-1} = 1$);
    \item except the multiplicative identity $0$, which has no multiplicative inverse (i.e. $0^{-1}$ is undefined);
    \end{itemize}
\end{itemize}

All fields of the same size are isomorphic, and therefore a field can be uniquely identified by its size. There are many ways to construct a field, with modular arithmetic being a common one. Finite fields are commonly constructed using integer arithmetic when $m=1$ (e.g. $\F_{13}$) or polynomial modular arithmetic otherwise (e.g. $\F_{2}[x]/(x^7+x^4+1)$.

Take for example the prime field $\F_{13}$ ($p=13,~m=1$), constructed as $(\{0, 1, 2, ... 11, 12\}, +, \cdot)$. In this field addition and multiplication are the usual operations, but always modulo the field size, $13$. Therefore:

$$5 + 10 \equiv 2 \mod 13 \\$$
$$5 - 10 \equiv 8 \mod 13 \\$$
$$5 \cdot 4 \equiv 7 \mod 13 \\$$

Division works by finding a value $A^{-1}$ such that $A \cdot A^{-1} \equiv 1 \mod p$. For example, $5^{-1} = 8 \mod 13$ because $5 \cdot 8 \equiv 1 \mod 13$.

For finite fields $\F_{p^m}$ with $m > 1$, one possible structure is to use polynomials of degree $<m$, and coefficientes in $\F_{p}$: $A = \sum_{i=0}^{m-1} a_i x^i,~a_i \in \F_{p}$. Then we choose an \emph{irreducible polynomial}, of degree $m$ and coefficientes in $\F_{p}$, to perform the modulo operation.

As an example, the field $\F_{2^7}$ ($p=2,~m=7$) can be constructed as:

$$\F_{2}[x]/(x^7+x^4+1) = (\{0, 1, x, x+1, x^2, x^2+1, x^2+x, ... x^6+x^5+x^4+x^3+x^2+x+1\}, +, \cdot)$$

In this case addition is performed as usual for polynomials, but with the coefficients in $\F_2$:

$$(x^2+x+1) + (x+1) = (x^2+2x+2) \equiv (x^2)$$.

Note that no modulo operation, or \emph{reduction}, is required for additions because they never surpass the degree of the operands. Multiplications, on the other hand, may increase the degree, and is performed modulo the irreducible polynomial $x^7+x^4+1$:

$$(x^4+1) \cdot (x^3+x^2+x+1) \equiv (x^6+x^5+x^3+x^2+x) \mod (x^7+x^4+1)$$.

The choice of irreducible polynomial doesn't affect the mathematical structure (it is still a $\F_{2^7}$ finite field), but it does affect the performance of operations in the field when inplemented in hardware or software.

In this work we focus on polynomial structures for finite fields, with $p=2$ unless noted. They are specially interesting for computations because the coefficients can be thought of as bits, and each element as an array of their coefficients:

\[
A = [a_{m-1}, a_{m-2}, ..., a_2, a_1, a_0] \\
\text{e.g.}~A(x) = x^6+x^3+x = 1x^6+0x^5+0x^4+1x^3+0x^2+1x+0 \rightarrow [1, 0, 0, 1, 0, 1, 0]
\].

In this view, a field addition is just a XOR operation item-wise:

$$(x^5+x^2)+(x^2+1) \rightarrow [0, 1, 0, 0, 1, 0, 0] \oplus [0, 0, 0, 0, 1, 0, 1] = [0, 1, 0, 0, 0, 0, 0]$$,

and a multiplication can be performed by a regular polynomial multiplication followed by a reduction function. This function takes a "double wide" polynomial (because the resulting degree can be up to $2m-2$), which is not an element of the field, and returns the representation of the correct element (the product):

\[
A(x) \cdot B(x) = C(x) \equiv D(x) \mod x^7+x^4+1\\
(x^6+1) \cdot (x^3+x^2+x+1) = x^9 + x^8 + x^7 + x^6 + x^3 + x^2 + x + 1 \\
\texttt{reduce}([0, 0, 0, 1, 1, 1, 1, 0, 0, 1, 1, 1, 1],~x^7+x^4+1) \rightarrow [0, 1, 0, 1, 1, 1, 1, 0]\\
\equiv x^6 + x^4 + x^3 + x^2 + x \mod x^7+x^4+1
\]

This function, $\texttt{reduce}(C, x^m+...+1) = D$ is the main focus of this work, as it is generalized and used for efficient exponentiations.