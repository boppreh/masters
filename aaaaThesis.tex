%%%%%%%%%%%%%%%%%%%%%%%%%%%%%%%%%%%%%%%%%%%%%%%%%%%%%%%%%%%%%%%%%%%%%%%
% Universidade Federal de Santa Catarina             
% Biblioteca Universitária                     
%                                                           
% (c)2010 Roberto Simoni (roberto.emc@gmail.com)
%         Carlos R Rocha (cticarlo@gmail.com)
%%%%%%%%%%%%%%%%%%%%%%%%%%%%%%%%%%%%%%%%%%%%%%%%%%%%%%%%%%%%%%%%%%%%%%%
%\PassOptionsToPackage{abnt-etal-cite=1, abnt-etal-list=0}{abntcite}
\documentclass{ufscThesis}

\usepackage[page]{appendix}

%\usepackage{cite}
\usepackage{amsmath}
\usepackage{amssymb}
\usepackage{amsfonts}
\usepackage[utf8]{inputenc}
\usepackage{algorithmic}
\usepackage{algorithm}% http://ctan.org/pkg/algorithms
\usepackage{multicol}
\usepackage{amsthm}

\usepackage{mathtools}
\DeclarePairedDelimiter{\ceil}{\lceil}{\rceil}
\DeclarePairedDelimiter{\floor}{\lfloor}{\rfloor}
\newcommand{\minuseq}{\mathrel{{-}{=}}}
\newcommand{\F}{\mathbb{F}}
\newcommand{\primes}{\mathbb{P}}
\newcommand{\weight}[1]{\text{weight}({#1})}

\renewcommand{\algorithmicrequire}{\textbf{Input:}}
\renewcommand{\algorithmicensure}{\textbf{Output:}}
\renewcommand{\algorithmicforall}{\textbf{for each}}

\newcommand{\mathbox}[3][l]{\makebox[\widthof{$#2$}][#1]{$#3$}}

% From Elsevier class.
\newtheorem{thm}{Theorem}
\newtheorem{lem}[thm]{Lemma}
%\newdefinition{rmk}{Remark}
%\newproof{pf}{Proof}
%\newproof{pot}{Proof}

\newcommand{\specialcell}[2][c]{%
  \begin{tabular}[#1]{@{}l@{}}#2\end{tabular}}

% Without this the superscripts get too close to the upper line.
\renewcommand{\arraystretch}{1.2}
\setlength\tabcolsep{5mm}

\usepackage [english]{babel}
\usepackage [autostyle, english = american]{csquotes}
\MakeOuterQuote{"}


\newcommand{\ftwo}{{\mathbb F}_{2}}
\newcommand{\ftwom}{{\mathbb F}_{2^m}}
\newcommand\Tstrut{\rule{0pt}{2.1ex}}  

\newcommand{\rmv}[1]{}

\newcommand*\BitAnd{\mathrel{\&}}
\newcommand*\BitOr{\mathrel{|}}
\newcommand*\BitXOR{\mathrel{\oplus}}
\newcommand*\ShiftLeft{\ll}
\newcommand*\ShiftRight{\gg}
\newcommand*\BitNeg{\ensuremath{\mathord{\sim}}}


%%%%%%%%%%%%%%%%%%%%%%%%%%%%%%%%%%%%%%%%%%%%%%%%%%%%%%%%%%%%%%%%%%%%%%%
% Identificadores do trabalho
% Usados para preencher os elementos pré-textuais
%%%%%%%%%%%%%%%%%%%%%%%%%%%%%%%%%%%%%%%%%%%%%%%%%%%%%%%%%%%%%%%%%%%%%%%
% TODO: M mínusculo
\titulo{Fast modular reduction \protect\\ and squaring in $GF(2^\MakeLowercase{m})$} % Titulo do trabalho
%\subtitulo{Estilo \LaTeX~ padrăo}                % Subtitulo do trabalho (opcional)
\autor{Lucas Boppre Niehues}           % Nome do autor
\data{01}{Janeiro}{2017}                           % Data da publicaçăo do trabalho

\orientador{Prof. Ricardo Felipe Custódio}                    % Nome do orientador e (opcional) seu título
\coorientador{Prof. Daniel Panario}                % Nome do coorientador e seu título (opcional)
\coordenador{Prof. Chefe, Carina Dorneles}              % Nome do coordenador do curso e (opcional) seu título

\instituicao[o]{Universidade Federal de Santa Catarina}
\departamento[a]{Departamento de Informática e Estatistica}
\curso[o]{curso de Pós-Graduação em Ciência da Computação}
\grau{Mestre em Ciência da Computação}
\documento[a]{Dissertação}

%%% Sobre a Banca
\numerodemembrosnabanca{2} % Isso decide se haverá uma folha adicional
%\orientadornabanca{sim} % Se faz parte da banca definir como sim
%\coorientadornabanca{sim} % Se faz parte da banca definir como sim
\bancaMembroA{Primeiro membro da banca} %Nome do presidente da banca %TODO
\bancaMembroB{Segundo membro da banca}      % Nome do membro da Banca %TODO

\textoResumo {Resumo em portugues} %TODO

\palavrasChave {Corpos finitos, teoria de números, polinômios, squaring, criptografia.}

\textAbstract {We present an efficient bit-parallel algorithm for squaring in $GF(2^m)$ using polynomial basis. This algorithm achieves competitive efficiency while being aimed at any choice of low-weight irreducible polynomial. For a large class of irreducible polynomials it is more efficient than the previously best general squarer. In contrast, other efficient squarers often require a change of basis or are suitable for only a small number of irreducible polynomials. Additionally, we present a simple algorithm for modular reduction with equivalent cost to the state of the art for general irreducible polynomials. This fast reduction is used in our squaring method.}

\palavrasChave {Finite field, number theory, polynomial, squaring, cryptography.}

%%%%%%%%%%%%%%%%%%%%%%%%%%%%%%%%%%%%%%%%%%%%%%%%%%%%%%%%%%%%%%%%%%%%%%%
% Início do documento                                
%%%%%%%%%%%%%%%%%%%%%%%%%%%%%%%%%%%%%%%%%%%%%%%%%%%%%%%%%%%%%%%%%%%%%%%
\begin{document}
%--------------------------------------------------------
% Elementos pré-textuais
\capa  
\folhaderosto % Se nao quiser imprimir a ficha, é só năo usar o parâmetro
\folhaaprovacao
\paginaresumo
\paginaabstract
\listadefiguras
\listadetabelas 
\listadeabreviaturas
\listadesimbolos
\sumario

%Dissertação segundo papel de 2016/07/01
%%%%%%%%%%%%%%%%%%%%%%%%%%%%%
%1. Introdução:
%contexto
%objetivo - específico
%justificativa
%motivação
%conteúdo (na seção x...) 
%
%2. ref. teórico / introdução
%o que o leitor precisa saber para entender a tua proposta
%
%3. Proposta / O que fez!
%Artigo
%
%4. Disc. / Análise / comparação
%Do que foi feito
%
%5. Considerações finais
%Apanhado do que foi feito, linkado com objetivos específicos
%trabalhos futuros
%
%referências
%%%%%%%%%%%%%%%%%%%%%%%%%%%%%
\chapter{Introduction}
% Contextualizar: do geral oara o específico, do que se trata o teu trabalho - 5 a 10 parágrafos contextualizando
% Qual é o problema, por que o problema é um problema
% Objetivos Geral e específicos
% Motivação
% Justificativa
% Metodologia de pesquisa adotada
% Opcional: listar as principais contribuicoes
% Opcional: Limitaçoes do teu trabalho, no sentido do que nao é o teu trabalho

% Motivação: por que trabalhar nesse tema. "trabalho de cooperação itnernacional labsec com a universidade de carleton, onde foram feitos vários trabalhos na área de processamento rápido de corpos finitos". aspeco social, pessoal, laboratório, pesquisa. "durante os anos as pesquisam sendo realizadas no grupo de pesquisas do labsec, chegou-se a conclusao que o desempenho e performance de algoritmos não era o desejado, com a vinda e internacionalização do grupo de pesquisa especial, e cooperação com a unviersidade de Carleton, descobriu-se que haviam alguams coisas a fazer..."
% Justificativa: científiac. É importante melhorar o desempenho pra trabalhar na aritmética em corpos finitos? Algoritmos são as peças chaves, diminuir úmero de operacóes leva a melhor performance...

% Näo incluir áreas de pesquisa que não deram resultado: shamir, ECC, etc. Se tiver alguma coisa na mesma área, incluir nos trabalhos futuros
Finite field arithmetic is the fundamental for many important cryptosystems, such as ECC (Elliptic Curve Cryptography). Such arithmetic is often implemented by performing operations on integers or polynomials, then performing modular reduction on the result. The usage of binary polynomials is specially common for hardware implementations, as the operations map well to logic gates.

The performance of such system is important because it maps directly to fabrication costs, circuit area, power consumption, and execution time. Since finite field arithmetic is the basis of various cryptosystems, any optimizations in this area will cause improvements in the general system. Such improvements are specially useful for constrained systems, such as sensor networks.

This work has two general objectives. First, to act as an introduction to finite field arithmetic, and how fields based on binary polynomials are used in practice, down to the logic gates. It is our hope that this introduction can help future researchers that wish to understand or improve this area.

Our second general objective is to propose improvements on finite field arithmetic, with the goal of having these low level improvements leading to improvements in the overall system. Specifically, reducing the delay and number of logic gates required for reduction and squaring in binary finite fields; generalizing these improvements to other finite fields implemented with polynomials; and finally, translating these low level improvements to software algorithms.

Our work was motivated by the international cooperation between our Universidade Federal de Santa Catarina, and the Canadian Carleton University. This cooperation produced many projects related to fast processing in finite fields, specifically for cryptographic systems. This choice of subject was the result of a general dissatisfaction of the speed of the high level systems. The approach of optimizing the low level components was seen as a natural consequence of the expertise brought by this cooperation.

In Chapter~\label{cap:background} we introduce the basic mathematics behind finite field, with a goal of both introducing the subject for interested researchers, and to give the necessary background for the subsequent chapters. We focus on reduction and exponentiation of binary polynomials, and propose new hardware algorithms in Chapter~\ref{chapters/reduction_squaring_2}. In Chapter~\ref{chapters/word_processing} we contextualize and provide tools on how to translate hardware algorithms in a way to also be efficient in software. Chapter~\ref{chapters/visual_debugger} showscases the visualization tools we developed to help this research. Chapter~\ref{chapters/final_considerations} contains our final considerations, giving a conclusion and raising interesting future works.


%\section{Objectives}
%
%General objectives:
%
%\begin{itemize}
%	\item Get introduced to finite field arithmetic.
%	\item Speed up cryptographic systems that depend on finite field arithmetic.
%\end{itemize}
%
%Specific objectives:
%
%\begin{itemize}
%	\item Create an efficient, general algorithm for binary polynomial reduction.
%	\item Create an efficient, general algorithm for binary polynomial squaring.
%	\item Translate speedups from hardware-oriented algorithms to software.
%\end{itemize}

% Introdução a finite fields
\chapter{Mathematical background}\label{cap:background}
\subsection{Finite fields}

A Finite Field $\F_{p^m}$ (also known as Galois Field $GF(p^m)$) is an abstract mathematical structure composed of a set of elements and two operators:

\[
\F_{p^m} = (\{A_0, A_1, A_2, ... A_{p^m-1}\}, +, \cdot)
\]

\begin{itemize}
\item $\F_{p^m}$: A finite field of size $p^m$, where:
    \begin{itemize}
    \item $p$: the \emph{field characteristic}, $p \in \mathbb{P}$ (i.e. must be a prime number);
    \item $m$: the \emph{extension size} $m \geq 1, m \in \mathbb{N}$;
    \end{itemize}
\item $\{A_0, A_1, A_2, ... A_{p^m-1}\}$: A set of $p^m$ \emph{field elements}. In this work we use the names $A$, $B$ and $C$ to refer to arbitrary members of this set, and $\F_{p^m}$ as an informal shorthand for this set (e.g. $A \in \F_{p^m}$).
\item $+, \cdot$~: Addition and multiplication operators, where:
    \begin{itemize}
    \item they are binary operators that operate field elements, $+, \cdot : (\F_{p^m} \times \F_{p^m}) \mapsto \F_{p^m}$;
    \item there are two elements labeled $0$ and $1$ that behave as the additive and multiplicative identities, $0 \neq 1$, such that $\forall A (A + 0 = A~\wedge~A \cdot 1 = A)$;
    \item both operations commute (i.e. $A + B = B + A$ and $A \cdot B = B \cdot A$) and associate (i.e. $(A + B) + C = A + (B + C)$ and $(A \cdot B) \cdot C = A \cdot (B \cdot C)$);
    \item every element has an additive and a multiplicative inverse, also in the field (i.e. there are elements $-A$ and $A^{-1}$ such that $A + -A = 0$ and $A \cdot A^{-1} = 1$);
    \item except the multiplicative identity $0$, which has no multiplicative inverse (i.e. $0^{-1}$ is undefined);
    \end{itemize}
\end{itemize}

All fields of the same size are isomorphic, and therefore a field can be uniquely identified by its size. There are many ways to construct a field, with modular arithmetic being a common one. Finite fields are commonly constructed using integer arithmetic when $m=1$ (e.g. $\F_{13}$) or polynomial modular arithmetic otherwise (e.g. $\F_{2}[x]/(x^7+x^4+1)$.

Take for example the prime field $\F_{13}$ ($p=13,~m=1$), constructed as $(\{0, 1, 2, ... 11, 12\}, +, \cdot)$. In this field addition and multiplication are the usual operations, but always modulo the field size, $13$. Therefore:

$$5 + 10 \equiv 2 \mod 13 \\$$
$$5 - 10 \equiv 8 \mod 13 \\$$
$$5 \cdot 4 \equiv 7 \mod 13 \\$$

Division works by finding a value $A^{-1}$ such that $A \cdot A^{-1} \equiv 1 \mod p$. For example, $5^{-1} = 8 \mod 13$ because $5 \cdot 8 \equiv 1 \mod 13$.

For finite fields $\F_{p^m}$ with $m > 1$, one possible structure is to use polynomials of degree $<m$, and coefficientes in $\F_{p}$: $A = \sum_{i=0}^{m-1} a_i x^i,~a_i \in \F_{p}$. Then we choose an \emph{irreducible polynomial}, of degree $m$ and coefficientes in $\F_{p}$, to perform the modulo operation.

As an example, the field $\F_{2^7}$ ($p=2,~m=7$) can be constructed as:

$$\F_{2}[x]/(x^7+x^4+1) = (\{0, 1, x, x+1, x^2, x^2+1, x^2+x, ... x^6+x^5+x^4+x^3+x^2+x+1\}, +, \cdot)$$

In this case addition is performed as usual for polynomials, but with the coefficients in $\F_2$:

$$(x^2+x+1) + (x+1) = (x^2+2x+2) \equiv (x^2)$$.

Note that no modulo operation, or \emph{reduction}, is required for additions because they never surpass the degree of the operands. Multiplications, on the other hand, may increase the degree, and is performed modulo the irreducible polynomial $x^7+x^4+1$:

$$(x^4+1) \cdot (x^3+x^2+x+1) \equiv (x^6+x^5+x^3+x^2+x) \mod (x^7+x^4+1)$$.

The choice of irreducible polynomial doesn't affect the mathematical structure (it is still a $\F_{2^7}$ finite field), but it does affect the performance of operations in the field when inplemented in hardware or software.

In this work we focus on polynomial structures for finite fields, with $p=2$ unless noted. They are specially interesting for computations because the coefficients can be thought of as bits, and each element as an array of their coefficients:

\[
A = [a_{m-1}, a_{m-2}, ..., a_2, a_1, a_0] \\
\text{e.g.}~A(x) = x^6+x^3+x = 1x^6+0x^5+0x^4+1x^3+0x^2+1x+0 \rightarrow [1, 0, 0, 1, 0, 1, 0]
\].

In this view, a field addition is just a XOR operation item-wise:

$$(x^5+x^2)+(x^2+1) \rightarrow [0, 1, 0, 0, 1, 0, 0] \oplus [0, 0, 0, 0, 1, 0, 1] = [0, 1, 0, 0, 0, 0, 0]$$,

and a multiplication can be performed by a regular polynomial multiplication followed by a reduction function. This function takes a "double wide" polynomial (because the resulting degree can be up to $2m-2$), which is not an element of the field, and returns the representation of the correct element (the product):

\[
A(x) \cdot B(x) = C(x) \equiv D(x) \mod x^7+x^4+1\\
(x^6+1) \cdot (x^3+x^2+x+1) = x^9 + x^8 + x^7 + x^6 + x^3 + x^2 + x + 1 \\
\texttt{reduce}([0, 0, 0, 1, 1, 1, 1, 0, 0, 1, 1, 1, 1],~x^7+x^4+1) \rightarrow [0, 1, 0, 1, 1, 1, 1, 0]\\
\equiv x^6 + x^4 + x^3 + x^2 + x \mod x^7+x^4+1
\]

This function, $\texttt{reduce}(C, x^m+...+1) = D$ is the main focus of this work, as it is generalized and used for efficient exponentiations.

% Artigo rejeitado para IEEE (https://www.sharelatex.com/project/56e84e3540deb4e636bf5aab)
\chapter{Reduction and Squaring in $GF(2^\MakeLowercase{m})$}
\section{Introduction}

Arithmetic in the finite field $GF(2^m)$ of $2^m$ elements (also denoted $\F_{2^m}$) is fundamental for many important cryptosystems such as ECC (Elliptic Curve Cryptography). Such arithmetic is usually implemented by choosing an irreducible polynomial $f \in \F_2[x]$, $\deg(f) = m$, performing operations and reducing modulo this polynomial. It is common to have algorithms that implement arithmetic operations using a particular class of irreducible polynomials. This is because these algorithms are efficiently implemented considering such class of irreducibles.\\

Arithmetic operations in $GF(2^m)$ usually consist of addition (being equivalent to subtraction in characteristic 2), multiplication (of which squaring is a special case), and multiplicative inverses, used in divisions. All these operations are used in ECC, making any optimizations reflect directly on the speed of elliptic curve arithmetic, raising the importance of choosing an irreducible polynomial and associated algorithms. As examples, the classes of pentanomials $x^m+x^{n+2}+x^{n+1}+x^{n}+1$, $x^{4s}+x^{3s}+x^{2s}+x^s+1$, $x^m+x^{m-r}+x^s+x^r+1$ and $x^m+x^{n+1}+x^n+x+1$ are used in fast multipliers~\cite{fan2015survey}, each with their own fast multiplication and reduction algorithms.\\

However, the algorithms designed for irreducible polynomials with specific exponents may contain internal structures with an unclear impact on security of applications that use these algorithms. No attacks have been demonstrated so far, but the security community has seen evidence of standards containing back doors~\cite{bernstein2016dual} and cryptography failing due to fixed parameters~\cite{adrian2015imperfect}. In light of these events there has been discussions for less magic parameters and more randomness in the structures used.\\

A related problem is that many classes of irreducible polynomials contain few elements with degrees interesting for instance for ECC, often having no irreducible polynomials at all for a desired degree. Furthermore, choosing a class to speed up a specific operation may lead to less efficient algorithms for other operations used in the same application. Choosing the parameters for a field is therefore an exercise in tradeoffs. \\

In this chapter we introduce a general algorithm for $GF(2^m)$ modular reduction and an efficient squarer suitable for \emph{any} low-weight irreducible polynomial $f$. These algorithms operate on elements represented in polynomial basis, where the coefficients are stored simply as an array of bits, and operations are performed bitwise in a logic circuit. Such algorithms are usually measured by the number of bit-level XOR operations performed and their circuit delay (we note that only XOR operations are required in these algorithms). Previous works on this operation have been limited to certain classes of irreducible polynomials to achieve competitive efficiency, fixing the weight (number of non-zero elements) and the relationship among exponents. 
Although our algorithms hold for any low-weight polynomial, to achieve better circuit delay and be more comparable to other proposals, we focus on the pentanomial case $x^m+x^a+x^b+x^c+1$, $m > a > b > c > 0$, where $a \leq \ceil{m/2}$.\\

Our squarer has different costs depending on the irreducible polynomial used. Applications requiring utmost efficiency should choose an irreducible polynomial to minimize the global cost of operations. We observe there are polynomials that minimize the number of XOR operations and delay with our squarer; we show that low-weight polynomials with this characteristic are abundant (see Table~\ref{table:comparison_squarer} at Section~\ref{comparison}). Additionally, our squarer can be used for higher weight polynomials, with some performance penalty.\\

The structure of the next sections is as follows. In Section \ref{modrec}, we present a general algorithm to perform modular reduction. This algorithm is generic and can be used with any irreducible polynomial. In Section \ref{squaring}, we modify the algorithm previously proposed to make it a squarer. This strategy allowed us to propose a squarer algorithm of low complexity. In Section \ref{comparison}, we compare our squarer with previous methods for this operation. Section~\ref{pth} shows how the reduction and squaring algorithm can be generalized for $p$-th power computation in characteristic $p$. \\

\section{Modular reduction} \label{modrec}

Let $GF(2^m)$ be a finite field generated by an irreducible polynomial $f(x) = x^m + r(x)$, where $\deg(r) < m$. Some polynomial basis operations performed on this field may require a reduction modulo $f$. This is a classical operation; see for example \cite[Chapter~2.3.5]{hankerson2006guide}.\\

Let $C(x) = \sum_{i=0}^{d} c_i x^i$, $d \geq m$, be the result of an operation before the reduction mod $f$ is executed. A simple method for reduction consists of iteratively computing $C(x) \leftarrow C(x) + c_{i} x^{i-m} f(x)$ for $i = d, d-1, \ldots, m$, where each step $i$ reduces the coefficient $c_{i}$.\\

To reduce the number of operations we replace $c_{i} x^{i-m} f(x)$ with $c_{i} x^{i-m} r(x)$ in the equation above, and then compute the final result mod $x^m$, which is a simple truncation. This avoids one XOR per iteration because $r$ has one less coefficient than $f$. We consider the truncation to have negligible cost. Algorithm~\ref{alg:reduce} is a pseudocode representation of these operations. The input consists of $d+1$ signals (the input coefficients), which are modified using XOR operations and returned as $m$ signals. We describe the details of this algorithm since it is a step in the construction of the main contribution in this chapter: the squarer method given in the next section. \\

\begin{algorithm}
\caption{General modular reduction for $GF(2^m)$}
\label{alg:reduce}
\begin{algorithmic}[1]
\REQUIRE $C = [c_0, c_1, c_2, ..., c_d]$, $d \geq m$, $f(x) = x^m + r(x)$
\ENSURE $C \mod f$
\FOR{$i = d, d-1, \ldots, m$}
\FORALL{exponent $e$ \textbf{of} $r$}
\STATE $C[i-m+e] \leftarrow C[i-m+e] \oplus C[i]$ \label{alg:reduce:op}
\ENDFOR
\ENDFOR
\RETURN $C[0],~C[1],~C[2],~\ldots,~C[m-1]$
\end{algorithmic}
\end{algorithm}

Algorithm~\ref{alg:reduce} uses $(d-m+1)w_r$ XOR operations, where $w_h$ is the weight of the polynomial $h$. If $C$ is a product of two polynomials of degree at most $m-1$, then the degree of $C$ is at most $d \leq 2m-2$. It is common, in practical implementations, to fix $d=2m-2$. In this case, the number of XOR operations to perform a modular reduction for trinomials and pentanomials are $2m-2$ and $4m-4$, respectively. The trinomial algorithm can be trivially modified to the equally spaced case, $x^m+x^{m/2}+1$, for $m$ even, where many operations cancel themselves out and result in $1.5 m - 1$ XOR operations~\cite{wu2002bit}.\\

The delay of Algorithm~\ref{alg:reduce} depends on the exact polynomial used, ranging from $2 T_X$ to $(m-1) T_X$, where $T_X$ is the delay of a single XOR gate. Unfortunately, the delay has no clear pattern in our algorithm. For the pentanomial case with $a \leq \ceil{m/2}$, we have empirically observed the following relation in irreducible pentanomials of degree up to 800, $m$ prime (see Table~\ref{table:reduce:delays} for statistical breakdown):

\[
    \text{delay}(a, b, c)= 
\begin{cases}
    6 T_X, & \text{if } c > \ceil{a/2}; \\
    5 T_X, & \text{if } \ceil{a/2} \geq c > \min(\ceil{b/2}, a-b+1); \\
    4 T_X, & \text{otherwise.}
\end{cases}
\]

\begin{table}
\centering
\caption{Circuit delay of Algorithm~\ref{alg:reduce} for irreducible pentanomials $x^m + x^a + x^b + x^c + 1$, $m < 800$, $m$ prime, $a \leq \ceil{m/2}$.}
{\begin{tabular}{l l l} \label{table:reduce:delays}
Delay & \# pentanomials & Percentage \\ \hline
$4 T_X$ & $694,789$ & $34.53\%$ \\ \hline
$5 T_X$ & $826,066$ & $41.06\%$ \\ \hline 
$6 T_X$ & $491,096$ & $24.41\%$ \\ \hline
All & $2,011,941$ & $100\%$
\end{tabular}}{}
\end{table}

Higher values of delay are observed when $a > \ceil{m/2}$. This trend can also be seen in other algorithms~\cite{fan2015survey}, and it is likely a consequence of the number of reduction steps~\cite{wu2002bit}. This means that for low delay reduction of products it is preferable to have irreducible polynomials where the second highest exponent is less than half of the degree $m$.\\

For the rest of this chapter we focus on pentanomials with $d=2m-2$ and $a \leq \ceil{m/2}$.

% \textbf{Algoritmo de redução de trinômios equally spaced?}

% \textbf{Texto sobre delays está comentado.}
%The bit operations are parallelizable, with a critical path of length $$\left \lfloor \frac{d-m}{m-\deg(r)} \right \rfloor + 1 .$$ The circuit delay is then the critical path length times the delay of a single XOR gate, $T_X$.

%When $d=m$ or $d=m+1$ the delay is always $1 T_X$, regardless of $\deg(r)$. For values of $d$ between $m+2 \leq d \leq 2m-2$ a delay of $1 T_X$ is only possible when $d + \deg(r) < 2m$. A delay of $1 T_X$ is not possible for $d > 2m-2$.

%We note that the delay is $2T_X$ if $$2m-d~\leq~\deg(r)~<~\frac{3m-d}{2},$$ and therefore only possible when $m+2 \leq d < 3m-2$.

% This particular case has been treated in the literature; a delay of $1 T_X$ is only achievable for irreducible trinomials $x^m+x+1$, and a delay of $2 T_X$ happens for any irreducible polynomial where $\deg(r) \leq \ceil{m/2}$.

%In practice $d$ is taken to be $2m-2$, the size of a product between two polynomials of degree $m-1$.

% \textbf{Mostrar dedução das desigualdades em um ou dois passos. "We can have other $T_X$". Mencionar que $2 T_X$ só é possível . "Portanto, existem muitos valores de $d$, dado $m$, tal que o polinômio $r(x)$ existe e o delay é $\leq 2 T_X$". "The particular case $d=2m-2$ has been treated in the literature, and in this case $r(x)$ must be..."}

\section{Squaring} \label{squaring}

Given an element $A(x) = \sum_{i=0}^{m-1} a_i x^i$ in $GF(2^m)$, computing its square is usually cheaper than naively multiplying $A$ by itself~\cite{fan2015survey}. For instance, using normal basis, a squarer is just a shift, but this basis is avoided in many situations due to the higher cost of multiplications using this type of  basis~\cite{fan2015survey}.\\

Algorithm~\ref{alg:reduce} can be modified to perform fast squaring operations in polynomial basis by setting $d=2m-2$ and using the property that $A^2 = \sum_{i=0}^{m-1} a_i x^{2i}$. Therefore, $A^2$ can be computed by interspacing zeroes between the coefficients of $A$, followed by a modular reduction, if necessary~\cite{wu2002bit}. The new algorithm is shown in Algorithm~\ref{alg:square}. The placing of zeroes is considered to have negligible cost on a logical circuit, hence the complexity of our squarer is restricted by the complexity of performing the modular reduction of $A^2$.\\

In Algorithm~\ref{alg:reduce} the operation $C[i-m+e] \leftarrow C[i-m+e] \oplus C[i]$ on Step~\ref{alg:reduce:op} performs one XOR and modifies one coefficient. When applied to $A^2$, each operation falls into one of three cases, depending on the parity of the indexes, and if the coefficients have been modified by a previous operation. In the following we analyze these three cases. \\

\begin{description}
\item[Case 1:] $i$ is odd and $C[i]$ has not yet been modified, therefore $C[i]=0$, and the operation has no effect.
    
\begin{lem} \label{lemma:case1}
Let $\beta$ be the highest exponent of $r$ such that $\beta \not\equiv m \pmod {2}$. Case 1 above allows for $(m-1-\beta) / 2$ coefficients to be skipped.
\end{lem}

\begin{proof}
The bits of $C$ are modified at the index $i-m+e$. The highest possible odd index for $i-m+e$ is $m-2+\beta$, when $i=2m-2$ (first iteration) and $e = \beta$. Therefore all bits at odd indexes greater than or equal to $m+\beta$ are known to be zero and are never modified (Case 1). This skips $\left((2m-2) - (m+\beta) + 1\right)/2 = (m-1-\beta)/2$ iterations. Additionally, these bits are never used and can be omitted from the circuit.
\end{proof}

\item[Case 2:] $i-m+e$ is odd and $C[i-m+e]$ has not yet been modified, therefore $C[i-m+e]=0$, no XOR is required, and only the copy $C[i-m+e] \leftarrow C[i]$ is necessary.
    \begin{lem} \label{lemma:case2}
Case 2 above allows for $(m-1+\beta)/2$ XOR operations to be replaced with a simpler copy operation.
\end{lem}

\begin{proof}
The first time an odd coefficient is modified, the XOR and assignment operation can be replaced with a simple assignment since the previous value is known to be zero (Case 2). There are $m-1$ odd coefficients, of which $(m-1-\beta)/2$ are skipped per Lemma~\ref{lemma:case1}. The total number of XOR operations avoided in this case is then $(m-1+\beta)/2$.
\end{proof}

\item[Case 3:] Otherwise, the operation continues normally and XORs are performed.
\end{description}

\vspace{2mm}

Using the savings provided by Lemma~\ref{lemma:case1} and Lemma~\ref{lemma:case2} we can give the total cost of squaring using Algorithm~\ref{alg:reduce}.

\begin{thm} \label{theorem}
Polynomial basis squaring in $GF(2^m)$ can be computed in $(w_r-1) (m-1+\beta)/2$ XOR operations.
\end{thm}

\begin{proof}
    Algorithm~\ref{alg:reduce} can be used to perform a modular reduction of $A^2$ in $(m-1)w_r$ XOR operations. Lemma~\ref{lemma:case1} allows for $(m-1-\beta)/2$ coefficients to be skipped, saving $w_r$ XORs each. Furthermore Lemma~\ref{lemma:case2} avoids another $(m-1+\beta)/2$ XOR operations. Hence we can reduce the number of XOR operations by a total of $w_r (m-1-\beta)/2 + (m-1+\beta)/2$. The total number of operations is then $$(m-1)w_r - w_r (m-1-\beta)/2 + (m-1+\beta)/2 = (w_r-1) (m-1+\beta)/2.$$ The interspacing of zeroes, simple assignments and final truncation are considered to have negligible cost.
\end{proof}

Algorithm~\ref{alg:square} performs the zero interspacing and uses the above cases to reduce the number of operations during the reduction. The internal variable $S$ keeps track of which coefficients have been assigned and therefore which states are applicable. The algorithm ends with a truncation mod $x^m$ as in Algorithm~\ref{alg:reduce}. We note the case analysis depends only on the irreducible polynomial $f$ and can be precomputed, resulting in a list of simple assignments and XOR operations with fixed indexes. \\

By Theorem~\ref{theorem}, the total number of XOR operations performed in Algorithm~\ref{alg:square} is $(w_r-1) (m-1+\beta)/2$. This is equivalent to the state of the art for trinomials~\cite{wu2002bit}, but we observe our algorithm holds for irreducible polynomials of any weight. When using irreducible pentanomials the cost ranges from $1.5(m-1)$ when $\beta=0$ (all exponents odd except the independent term) to $3(m-1)$ when $\beta=m-1$. \\

For irreducible septanomials the cost of squaring is $2.5(m-1+\beta)$ XORs. This is interesting as a low $\beta$ septanomial can be more efficient than a high $\beta$ pentanomial. Unfortunately, there has been little research for septanomials and higher weights, therefore we focus on pentanomials for this chapter. \\

Having $\beta=0$ implies $m$, $a$, $b$ and $c$ are odd, and such irreducible pentanomials are plentiful. If the exponents of the pentanomial were randomly distributed, it is expected that about $12.5\%$ of the odd degree pentanomials would have $\beta=0$. We enumerate all irreducible binary pentanomials with $m < 800$, $m$ a prime number (a common assumption in ECC \cite{doche2005redundant}), $a \leq \ceil{m/2}$ (for low delay), and observe $2,011,941$ meet this restriction. \\

\begin{algorithm}
\caption{General squaring for $GF(2^m)$}
\label{alg:square}
\begin{algorithmic}[1]
\REQUIRE $A = [a_0, a_1, a_2, ..., a_{m-1}]$, $f(x) = x^m + r(x)$
\ENSURE $A^2 \mod f$

\STATE $C = [a_0, 0, a_1, 0, a_2, 0, \ldots, 0, a_{m-1}]$ \COMMENT{interspace zeroes}
\STATE $S = [0, 0, 0, \ldots, 0, 0]$, $|S| = |C|$

\FOR{$i = 2m-2, 2m-3, \ldots, m$}
    \IF{$i$ is odd \textbf{and} $S[i]=0$}
        \STATE{do nothing} \COMMENT{Case 1}
    \ELSE
        \FORALL{exponent $e$ \textbf{of} $r$}
            \IF{$i-m+e $ is odd \textbf{and} $S[i-m+e]=0$}
                \STATE{$C[i-m+e] \leftarrow C[i]$} \COMMENT{Case 2}
            \ELSE
                \STATE $C[i-m+e] \leftarrow C[i-m+e] \oplus C[i]$ \COMMENT{Case 3}
            \ENDIF
            \STATE $S[i-m+e] = 1$
        \ENDFOR
    \ENDIF
\ENDFOR

\RETURN $C[0],~C[1],~C[2],~\ldots,~C[m-1]$
\end{algorithmic}
\end{algorithm}

The squarer inherits the delay characteristics of the reduction algorithm and therefore must be evaluated on a per-case basis, but is always equivalent or better than the delay for a full reduction. Based on our observations we conjecture that the delay is between $3 T_X$ and $6 T_X$ for pentanomials when $a \leq \ceil{m/2}$ (see Table~\ref{table:square:delays}). \\

\begin{table}
\centering
\caption{Circuit delay of Algorithm~\ref{alg:square} for irreducible pentanomials $x^m + x^a + x^b + x^c + 1$, $m < 800$, $m$ prime, $a \leq \ceil{m/2}$.}
{\begin{tabular}{l l l} \label{table:square:delays}
Delay & \# pentanomials & Percentage \\ \hline
$3 T_X$ & $475,528$ & $23.63\%$ \\ \hline
$4 T_X$ & $1,076,788$ & $53.52\%$ \\ \hline
$5 T_X$ & $403,801$ & $20.07\%$ \\ \hline 
$6 T_X$ & $55,824$ & $2.78\%$ \\ \hline
All & $2,011,941$ & $100\%$
\end{tabular}}{}
\end{table}

\section{Comparison} \label{comparison}

There have been numerous works on the choice of irreducible polynomials for binary fields $GF(2^m)$ with the goal of optimizing bit-level arithmetic operations \cite{fan2015survey}. Trinomials and equally spaced polynomials are generally preferred but not available for every field size. \\

Table~\ref{table:comparison_squarer} compares the XOR count and delay of the best $GF(2^m)$ squarers in the literature that operate at the bit level and support polynomial basis. As noted, squarers from the literature require the irreducible polynomial $f$ to adhere to certain restrictions. For comparison we enumerate all pentanomials $f(x) = x^m + x^a + x^b + x^c + 1,~m < 800$, prime $m$, and we note how many of those adhered to each method's restrictions. This is a measure of the squarer's generality, as ideally a squarer should work on as many different irreducible polynomials as possible. \\

The current state of the art for polynomial basis squaring of a generic pentanomial is Park~\cite{park2012explicit}. The number of XOR gates is given by an upper bound of $(3m+7a-b-3c+25)/2$, and circuit delay at most $3T_X$. We note our squarer has an exact formula for the number of XOR gates, and it is strictly smaller than the upper bound mentioned. This can be seen by noting that $\beta \in \{a, b,c, 0\}$, thus our worst case is when $\beta = a$ resulting in $(3m+3a-3)/2$ XORs. Meanwhile, the best case for Park's upper bound happens when $b=a-1$ and $c=a-2$, resulting in $(3m+3a+32)/2$ XORs. We note that our worst case has a strictly smaller number of XOR operations than the current best upper bound. \\

\begin{table}
\centering
\caption{$GF(2^m)$ squarers for $x^m + x^a + x^b + x^c + 1$ in polynomial basis, $m > a > b > c > 0$, $m < 800$, $m$ prime, $\beta = \max{\{a, b, c, 0\}}$ such that $\beta \not \equiv m \pmod{2}$. Compare to results from Hariri \cite{hariri2009bit} and Park \cite{park2012explicit}}
{\begin{tabular}{l r l}
\label{table:comparison_squarer}
Proposal / Polynomials & \# irreducible & XOR count / Delay  \\ \hline
\specialcell{Hariri \\ $a=b+1=c+2$} & \specialcell{$602$} & \specialcell{$(m-3)/2 + m + 4$ \\ $2T_X$} \\ \hline
\specialcell{Park \\ $a \leq \ceil{m/2}$} & \specialcell{$2,011,941$}& \specialcell{$\leq (3m+7a-b-3c+25)/2$ \\ $3T_X$} \\ \hline
\specialcell{\emph{Ours, $\beta=0$, $3 T_X$ delay} \\ $a \leq \ceil{m/2}$} & \specialcell{$2,690$} & \specialcell{$1.5(m-1)$ \\ $3 T_X$} \\ \hline
\specialcell{\emph{Ours, any $\beta$, $3 T_X$ delay} \\ $a \leq \ceil{m/2}$} & \specialcell{$475,528$} & \specialcell{$1.5(m-1+\beta)$ \\ $3 T_X$} \\ \hline
\specialcell{\emph{Ours, $\beta=0$, any delay} \\ $a \leq \ceil{m/2}$} & \specialcell{$233,974$} & \specialcell{$1.5(m-1)$ \\ $3 T_X \leq k \leq 6 T_X$} \\ \hline
\specialcell{\emph{Ours, any $\beta$, any delay} \\ $a \leq \ceil{m/2}$} & \specialcell{$2,011,941$} & \specialcell{$1.5(m-1+\beta)$ \\ $3 T_X \leq k \leq 6 T_X$} \\ \hline
\specialcell{\emph{Ours, any $\beta$, any delay} \\ any $a<m$} & \specialcell{all} & \specialcell{ $1.5(m-1+\beta)$ \\ $3 T_X \leq k \leq (m-1) T_X$}
\end{tabular}}{}
\end{table}

Our squarer has variable delay depending on the irreducible pentanomial used, with values between $3 T_X$ to $6 T_X$ observed for $m < 800$, $m$ prime, $a \leq \ceil{m/2}$ (see Table~\ref{table:square:delays} for statistical distribution). Additionally, one can manually tweak the resulting circuit in order to reduce the total time delay, as shown in Appendix~\ref{appendix:example}. \\

Table~\ref{table:comparison_nist} compares the cost of using a naive reduction squaring (such as interspacing zeroes and applying Algorithm~\ref{alg:reduce}), with Park's squarer~\cite{park2012explicit} and our squarer (Algorithm~\ref{alg:square}) when applied to NIST ECC irreducible pentanomials. The squarer from Hariri~\cite{hariri2009bit} is not applicable in this case as NIST pentanomials are not of the format $x^m+x^{n+2}+x^{n+1}+x^{n}+1$. \\ 

\begin{table}
\centering
\smaller
\caption{Squarer cost for NIST suggested ECC irreducible pentanomials, in number of XOR operations and circuit delay, where "naive reduction" is Algorithm~\ref{alg:reduce}.}
{\begin{tabular}{l r r r} \label{table:comparison_nist}
Polynomial & Naive reduction & Park \cite{park2012explicit} & Our squarer \\ \hline
$x^{163} + x^7 + x^6 + x^3 + 1$ & $648\oplus/~5T_X$ & $246\oplus/~3T_X$ & $252\oplus/~4T_X$ \\ \hline
$x^{283} + x^{12} + x^7 + x^5 + 1$ & $1,118\oplus/~5T_X$ & $\leq468\oplus/~\leq3T_X$ & $441\oplus/~4T_X$ \\ \hline\
$x^{571} + x^{10} + x^5 + x^2 + 1$ & $2,280\oplus/~4T_X$ & $\leq898\oplus/~\leq3T_X$ & $870\oplus/~3T_X$
\end{tabular}}{}
\end{table}

\section{Generalized reduction and squaring in $GF(\MakeLowercase{p}^\MakeLowercase{m})$} \label{pth}

We recall that the reduction can be performed by iteratively computing $C(x) \leftarrow C(x) + c_{i} x^{i-m} f(x)$ for $i = d, d-1, \ldots, m$. This is sufficient to reduce polynomials in binary fields, but can be easily generalized to other finite field characteristics. \\

Let $C$ and $F$ be polynomials with coefficients in $\F_p$, $p$ a prime number, where $C = \sum_{i=0}^d c_i x^i$, and $F(x) = \sum_{i=0}^m f_i x^i$; we want to compute $C \mod F$. The $i$-th term of $C$ can be reduced by subtracting $c_i f_m^{-1} x^{i-m} F(x)$. Thus a full reduction can be performed by computing

$$C(x) \leftarrow C(x) - c_i f_m^{-1} x^{i-m} F(x).$$

There are two optimizations available. First, we note that after cancelling a power we do not use that array slot any more. Therefore we do not need to actually clear that coefficient but only truncate the result at the end. Second, the inverse $z = f_m^{-1}$ can be precomputed. The optimized algorithm is presented in Algorithm \ref{alg:general:reduce}. The total cost is $(d-m+1) (w_F-1)$ subtractions and twice as many multiplications, all in the characteristic of the finite field. \\

\begin{algorithm}
\caption{Optimized algorithm for calculating $C \mod F$}
\label{alg:general:reduce}
\begin{algorithmic}[1]
    \REQUIRE $C = [c_0, c_1, c_2, ..., c_d]$, $d \geq m$, $F(x) = f_m x^m + r(x)$, $z = f_m^{-1}$
    \ENSURE $C \mod F$
    \FOR{$i = d, d-1, \ldots, m$}
        \FORALL{exponent $e$ \textbf{of} $r$}
            \STATE $C[i-m+e] \leftarrow C[i-m+e] - C[i] \cdot z \cdot f_e \mod p$
        \ENDFOR
    \ENDFOR
    \RETURN $C[0],~C[1],~C[2],~\ldots,~C[m-1]$
\end{algorithmic}
\end{algorithm}

We note that in $\F_2$ subtractions are XORs; furthermore $z$ and $f_e$ are 1 and the multiplication can be removed, resulting in the $\F_2$ reduction Algorithm~\ref{alg:reduce}. Analogous to the squaring properties and case analysis presented in Section~\ref{squaring} we can use the general reduction performed by Algorithm~\ref{alg:general:reduce} to compute $p$-th powers. \\

The squaring property of $\F_2$ can be generalized to $A^p = \sum_{i=0}^{m-1} a_i x^{pi}$ for fields $\F_{p^m}$. Therefore to compute the $p$-th power of a polynomial in polynomial basis we can interspace $p-1$ zeros between them and perform a reduction. The case analysis of Section~\ref{squaring} can be rewritten to the $\F_{p^m}$ case as follows: \\

\begin{description}
\item[Case 1]: $i \not\equiv 0 \pmod{p}$ and $C[i]$ has not been modified, therefore $C[i] = 0$, and the operation has no effect;
\item[Case 2]: $i-m+e \not\equiv 0 \pmod{p}$ and $C[i-m+e]$ has not been modified, therefore $C[i-m+e] = 0$, the left side of the subtraction is $0$, and may be easily computed;
\item[Case 3]: otherwise, the operation continues normally and one subtraction and two multiplications are performed in the prime field $\F_p$.
\end{description}
    
The number of operations saved follows a similar pattern to our squarer but with more cancellations due to the extra zeroes inserted. It is not easy to derive a formula for the exact number of operations and resulting delay; if needed, both can be computed for specific cases by running the algorithm over the desired irreducible polynomial. \\

% Parte de p-ésima potência do artigo submetido por último (https://www.sharelatex.com/project/5797e2f474f358335fdc55f5)
\chapter{Generalized reduction and squaring in $GF(\MakeLowercase{p}^\MakeLowercase{m})$} \label{pth}
We recall that the reduction can be performed by iteratively computing $C(x) \leftarrow C(x) + c_{i} x^{i-m} f(x)$ for $i = d, d-1, \ldots, m$. This is sufficient to reduce polynomials in binary fields, but can be easily generalized to other finite field characteristics.

\begin{table}
\centering
\caption{Squarer cost for NIST suggested ECC irreducible pentanomials, in number of XOR operations and circuit delay, where "naive reduction" is Algorithm~\ref{alg:reduce}.}
{\begin{tabular}{l r r r} \label{table:comparison_nist}
Polynomial & Naive reduction & Park \cite{park2012explicit} & Our squarer \\ \hline
$x^{163} + x^7 + x^6 + x^3 + 1$ & $648\oplus/~5T_X$ & $246\oplus/~3T_X$ & $252\oplus/~4T_X$ \\ \hline
$x^{283} + x^{12} + x^7 + x^5 + 1$ & $1,118\oplus/~5T_X$ & $\leq468\oplus/~\leq3T_X$ & $441\oplus/~4T_X$ \\ \hline
$x^{571} + x^{10} + x^5 + x^2 + 1$ & $2,280\oplus/~4T_X$ & $\leq898\oplus/~\leq3T_X$ & $870\oplus/~3T_X$
\end{tabular}}{}
\end{table}

Let $C$ and $F$ be polynomials with coefficients in $\F_p$, $p$ a prime number, where $C = \sum_{i=0}^d c_i x^i$, and $F(x) = \sum_{i=0}^m f_i x^i$; we want to compute $C \mod F$. The $i$-th term of $C$ can be reduced by subtracting $c_i f_m^{-1} x^{i-m} F(x)$. Thus a full reduction can be performed by computing

$$C(x) \leftarrow C(x) - c_i f_m^{-1} x^{i-m} F(x).$$

There are two optimizations available. First, we note that after cancelling a power we do not use that array slot any more. Therefore we do not need to actually clear that coefficient but only truncate the result at the end. Second, the inverse $z = f_m^{-1}$ can be precomputed. The optimized algorithm is presented in Algorithm \ref{alg:general:reduce}. The total cost is $(d-m+1) (w_F-1)$ subtractions and twice as many multiplications, all in the characteristic of the finite field.

\begin{algorithm}
\caption{Optimized algorithm for calculating $C \mod F$}
\label{alg:general:reduce}
\begin{algorithmic}[1]
    \REQUIRE $C = [c_0, c_1, c_2, ..., c_d]$, $d \geq m$, $F(x) = f_m x^m + r(x)$, $z = f_m^{-1}$
    \ENSURE $C \mod F$
    \FOR{$i = d, d-1, \ldots, m$}
        \FORALL{exponent $e$ \textbf{of} $r$}
            \STATE $C[i-m+e] \leftarrow C[i-m+e] - C[i] \cdot z \cdot f_e \mod p$
        \ENDFOR
    \ENDFOR
    \RETURN $C[0],~C[1],~C[2],~\ldots,~C[m-1]$
\end{algorithmic}
\end{algorithm}

We note that in $\F_2$ subtractions are XORs; furthermore $z$ and $f_e$ are 1 and the multiplication can be removed, resulting in the $\F_2$ reduction Algorithm~\ref{alg:reduce}. Analogous to the squaring properties and case analysis presented in Section~\ref{squaring} we can use the general reduction performed by Algorithm~\ref{alg:general:reduce} to compute $p$-th powers.

The squaring property of $\F_2$ can be generalized to $A^p = \sum_{i=0}^{m-1} a_i x^{pi}$ for fields $\F_{p^m}$. Therefore to compute the $p$-th power of a polynomial in polynomial basis we can interspace $p-1$ zeros between them and perform a reduction. The case analysis of Section~\ref{squaring} can be rewritten to the $\F_{p^m}$ case as follows:

\begin{description}
\item[Case 1]: $i \not\equiv 0 \pmod{p}$ and $C[i]$ has not been modified, therefore $C[i] = 0$, and the operation has no effect;
\item[Case 2]: $i-m+e \not\equiv 0 \pmod{p}$ and $C[i-m+e]$ has not been modified, therefore $C[i-m+e] = 0$, the left side of the subtraction is $0$, and may be easily computed;
\item[Case 3]: otherwise, the operation continues normally and one subtraction and two multiplications are performed in the prime field $\F_p$.
\end{description}
    
The number of operations saved follows a similar pattern to our squarer but with more cancellations due to the extra zeroes inserted. It is not easy to derive a formula for the exact number of operations and resulting delay; if needed, both can be computed for specific cases by running the algorithm over the desired irreducible polynomial.

% Technical report (https://www.sharelatex.com/project/55f7e238e04e2805561c2a10)
\chapter{Word Processing}
During our research we also studied $GF(2^m)$ arithmetic for CPUs, where the coefficient bits are grouped into words. Most of the results were about polynomial reduction modulo a trinomial, a common operation in $\F_2[x]/(x^m+x^a+1)$. This type of field is interesting because it maximizes the number of zero coefficients in the the irreducible polynomial, enabling numerous simplifications. In turn, the reduction operation is used in multiplications and exponentiations, making speedups desirable.

Performing this operation in a CPU has a few differences from implementing a hardware circuit. For example, circuits often avoid irreducible polynomials with $a > m/2$. This is caused by the number of reduction steps in the standard technique, whcih represents inter-dependency of the calculations, and thus increases the maximum delay in a circuit. The maximum number $k$ of reduction steps for a trinomial 
$x^{m} + x^{a} + 1$ in terms of the exponent 
$a$ is given by Sunar and Ko\c{c} \cite{sunar1999mastrovito}
\begin{equation} \label{eq:k}
  k = \left \lfloor \frac{m-2}{m-a} \right \rfloor + 1.
\end{equation}

However, the number of reduction steps does not affect software implementations. To illustrate this, we take two algorithms that perform modular reduction for the NIST polynomials $x^{233} + x^{74} + 1$ and $x^{409} + x^{87} + 1$~\cite[p. 55]{hankerson2006guide}. Then we manually adapt each of these algorithms to perform a reduction modulo its reciprocal $x^m + x^{a-m} + 1$. Notice the only change required is in the indexing and bitwise shifts. The total number of operations remains the same. Algorithms meant for circuit implementations, on the other hand, would have different delay characteristics when 

To illustrate this, consider the following four CPU-targeted reduction algorithms: $x^{233} + x^{74} + 1$ and $x^{409} + x^{87} + 1$~\cite[p. 55]{hankerson2006guide}, and our adaptations. The second algorithm is our adaption to perform a polynomial reduction modulus $x^{233} + x^{159} + 1$, its reciprocal. Notice they perform exactly the same number of operations, but would have drastically different performance if implemented in hardware.


\begin{algorithm}
\begin{algorithmic}[1]
  \REQUIRE $C[2m-2,0]$
  \ENSURE $C[m-1,0]$
  \FOR{$i \gets 15$ \textbf{downto} $8$}
    \STATE $T \gets C[i]$
    \STATE $C[i-8] \gets C[i-8] \oplus T << 23$
    \STATE $C[i-7] \gets C[i-7] \oplus T >> 9$
    \STATE $C[i-5] \gets C[i-5] \oplus T << 1$
    \STATE $C[i-4] \gets C[i-4] \oplus T >> 31$
  \ENDFOR
  \STATE $T \gets C[7] >> 9$
  \STATE $C[0] \gets C[0] \oplus T$
  \STATE $C[2] \gets C[2] \oplus T << 10$
  \STATE $C[3] \gets C[3] \oplus T >> 22$
  \STATE $C[7] \gets C[7] \& \texttt{0x1FF}$
  \RETURN $C$
  \caption{Hankerson's algorithm for reduction modulus $x^{233} + x^{74} + 1$, a standardized NIST polynomial.}
  \label{alg:233_74_nist}
\end{algorithmic}
\end{algorithm}

 \begin{algorithm}
 \begin{algorithmic}[1]
  \REQUIRE $C[2m-2,0]$
  \ENSURE $C[m-1,0]$
  \FOR{$i \gets 14$ \textbf{downto} $8$}
    \STATE $T \gets C[i]$
    \STATE $C[i-8] \gets C[i-8] \oplus T << 23$
    \STATE $C[i-7] \gets C[i-7] \oplus T >> 9$
    \STATE $C[i-3] \gets C[i-3] \oplus T << 22$
    \STATE $C[i-2] \gets C[i-2] \oplus T >> 10$
  \ENDFOR
  \STATE $T \gets C[7] >> 9$
  \STATE $C[0] \gets C[0] \oplus T$
  \STATE $C[2] \gets C[2] \oplus T << 31$
  \STATE $C[3] \gets C[3] \oplus T >> 1$
  \STATE $C[7] \gets C[7] \& \texttt{0x1FF}$
  \RETURN $C$
  \caption{Algorithm for reduction modulus $x^{233} + x^{159} + 1$, $(233, 74)$'s recriprocal.}
  \label{alg:233_159}
\end{algorithmic}
\end{algorithm}

\begin{algorithm}
\begin{algorithmic}[1]
  \REQUIRE $C[2m-2,0]$
  \ENSURE $C[m-1,0]$
  \FOR{$i \gets 25$ \textbf{downto} $13$}
    \STATE $T \gets C[i]$
    \STATE $C[i-13] \gets C[i-13] \oplus T << 7$
    \STATE $C[i-12] \gets C[i-12] \oplus T >> 25$
    \STATE $C[i-11] \gets C[i-11] \oplus T << 30$
    \STATE $C[i-10] \gets C[i-10] \oplus T >> 2$
  \ENDFOR
  \STATE $T \gets C[12] >> 25$
  \STATE $C[0] \gets C[0] \oplus T$
  \STATE $C[2] \gets C[2] \oplus T << 23$
  \STATE $C[12] \gets C[12] \& \texttt{0x1FFFFFF}$
  \RETURN $C$
  \caption{Hankerson's algorithm for reduction modulus $x^{409} + x^{87} + 1$, a standardized NIST polynomial.}
  \label{alg:409_87_nist}
\end{algorithmic}
\end{algorithm}


\begin{algorithm}
\begin{algorithmic}[1]
  \REQUIRE $C[2m-2,0]$
  \ENSURE $C[m-1,0]$
  \FOR{$i \gets 25$ \textbf{downto} $13$}
    \STATE $T \gets C[i]$
    \STATE $C[i-13] \gets C[i-13] \oplus T << 7$
    \STATE $C[i-12] \gets C[i-12] \oplus T >> 25$
    \STATE $C[i-3] \gets C[i-3] \oplus T << 9$
    \STATE $C[i-2] \gets C[i-2] \oplus T >> 23$
  \ENDFOR
  \STATE $T \gets C[12] >> 25$
  \STATE $C[0] \gets C[0] \oplus T$
  \STATE $C[10] \gets C[10] \oplus T << 2$
  \STATE $C[12] \gets C[12] \& \texttt{0x1FFFFFF}$
  \RETURN $C$
  \caption{Algorithm for reduction modulus $x^{409} + x^{322} + 1$, $(409, 87)$'s reciprocal.}
  \label{alg:409_322}
\end{algorithmic}
\end{algorithm}

\section{Choice of $a$}

The choice of the second coefficient $a$ directly affects the performance characteristics of reduction algorithms. Several factors are at play:

\begin{itemize}
\item The number of reduction steps depends on the proximity of $a$ and $m$. The number of steps increases with $a$, in discrete steps.
\item The number of reduction steps affects the depth of the circuit when implemented in hardware, though usually in logarithm fashion.
\item If $m-a < W$ software implementations will be less efficient. This is because there will be XOR operations inside words, and the operations can not be performed with full word-sized XORs.
\item The alignment of $a$ and $m-a$ with relation to $W$ define if word-sized XORs will require bits of two words in the source or in the destination. Thus alignment can halve the number of operations involved.
\end{itemize}

\section{Existing algorithms}\label{existing-algorithms}

Masoleh \cite[p. 953]{Masoleh} presents a recursive multiplier algorithm with $N_\oplus = m^2-1$ for $x^m + x^a + 1$ and $N_\oplus = m^2-\frac{m}{2}$ if $a = \frac{m}{2}$. If we consider that the multiplication takes ${(m-1)}^2$ XORs, then, the number of XORs for the reduction is $2m-2$ and $\frac{3}{2}m-1$, respectively. 

{[}NIST, Scott, etc. just dry descriptions{]}

\section{Reduction equation}

[This is an attempt to formulate the reduction operation as an equation. This is useful for proving correctness of algorithms by showing equivalence to the equations. I'm not 100\ they are correct. (I think they are not truncating the value to the modulus degree, but that should be easy to fix)]

General polynomial reduction equation with coefficients in $GF(2)$, with reduction steps in evidence:

\begin{gather}
    \sum_{i = 0}^{2m - 2}{c_i x^i} \mod \sum_{i = 0}^{m}{k_i x^i} = \\
    \sum_{i = 0}^{m-1}{c_i x^i} + \sum_{i=0}^{\floor*{\frac{m - 2}{m - a}}}
        \sum_{l=0}^{m}{ \sum_{j = 0}^{m - 2 - ia}{k_l c_{j + m + ia} x^{j+l}}  }
\end{gather}

General trinomial reduction equation ($1 \le a < m$), with reduction steps in evidence:

\begin{gather}
    \sum_{i = 0}^{2m - 2}{c_i x^i} \mod x^m + x^a + 1 = \\
    \sum_{i = 0}^{m-1}{c_i x^i} + \sum_{i=0}^{\floor*{\frac{m - 2}{m - a}}} \left(
        \sum_{j = 0}^{m - 2 - ia}{c_{j + m + ia} x^j} +
        \sum_{j = 0}^{m - 2 - ia}{c_{j + m + ia} x^{j+a}}
    \right) 
\end{gather}

Special case $a=1$ (single step):

\begin{gather}
    \sum_{i = 0}^{m-1}{c_i x^i} +
        \sum_{j = 0}^{m - 2}{c_{j + m} x^j} +
        \sum_{j = 0}^{m - 2}{c_{j + m} x^{j+a}}
\end{gather}

Special case $1 < a < \frac{m}{2}$ (two steps):

\begin{gather}
    \sum_{i = 0}^{m-1}{c_i x^i} +
        \sum_{j = 0}^{m - 2}{c_{j + m} x^j} +
        \sum_{j = 0}^{m - 2}{c_{j + m} x^{j+a}} +
        \sum_{j = 0}^{m - 2 - a}{c_{j + m + a} x^j} +
        \sum_{j = 0}^{m - 2 - a}{c_{j + m + a} x^{j+a}}
\end{gather}

Special case $a = \frac{m}{2}$ (two steps with cancellation):

\begin{gather}
    \sum_{i = 0}^{m-1}{c_i x^i} +
        \sum_{j = 0}^{\frac{m}{2} - 2}{c_{j + m} x^j} +
        \sum_{j = 0}^{m - 2}{c_{j + m} x^{j+\frac{m}{2}}} +
        \sum_{j = 0}^{\frac{m}{2} - 2}{c_{j + m + a} x^j}
\end{gather}


\section{Operating on words}\label{operating-on-words}

Bit manipulation operations are not efficiently supported in most programming languages. Programming languages, in general, are word oriented following a specific microprocessor architecture. For instance, numbers with $32$- or $64$-bit word size. The reduction algorithms themselves are strongly bit oriented. The codification of these algorithms to word oriented programming languages usually add some complexity in terms of new operations. Specific techniques that help us in this coding have been proposed in the literature\cite{Hilewitz2008}. 

Interoperability often means an irreducible polynomial must behave well
in both hardware and software implementations. A straightforward
approach to this problem is to develop a hardware implementation first,
XOR'ing individual bits, and convert the algorithm to words. This can be
thought of as parallelizing the operations, with SHIFTs and AND/OR masks
for alignment. This is the approach used in this document. {[}more
formalization? references of other similar approaches?{]}

The difficulty of the conversion process greatly varies depending on the
access pattern of the algorithm and alignment of words. If all coefficients inside a range are
XOR'ed once in a linear fashion, the XOR distance a multiple of the word size, and the start and end range lie in word boundaries, the converted algorithm gains a full
speedup of up to \texttt{WORD\_SIZE} times
{[}$C_{word} = \ceil*{\frac{C_{bit}}{W}}${]}.

\section{Bit operation reduction algorithms}\label{sec:bit:operation}
This section presents bit operation reduction algorithms for trinomials. The algorithms depend on the number of reduction steps given by Eq.~\ref{eq:k}. As $k$ increases, more steps can be required to perform the reduction. The equation $x^{m+j} \equiv x^{a+j} + x^j mod (x^m+x^a+1)$ is used to replace each coefficient at or over $m$ with a smaller one, reducing the polynomial. The Algorithm \ref{alg:bits} shows this reduction process. 

\begin{algorithm}
\begin{algorithmic}[1]
  \REQUIRE $a$, $C[2m-2,0]$
  \ENSURE $C[m-1,0]$
  \STATE $k \gets \left \lfloor \frac{m-2}{m-a} \right \rfloor + 1$
  \STATE $r \gets 0$
  \FOR{$i \gets 1$ \textbf{to} $k$}
    \STATE $r \gets r + (m-a)$
    \FOR{$j \gets 0$ \textbf{to} $m-1-r$}
      \STATE $C[j] \gets C[j] \oplus C[j+a+r]$
      \ENDFOR
    \FOR{$j \gets a$ \textbf{to} $m-1$}
      \STATE $C[j] \gets C[j] \oplus C[j+r]$
      \ENDFOR
    \ENDFOR
  \RETURN $C$
  \caption{General reduction algorithm processed by bit, taken directly from equation 5 (update).}
  \label{alg:bits}
\end{algorithmic}
\end{algorithm}

However this algorithm introduces a number of repeated bit operations. Some of these operations could be canceled, and others could be grouped, so that they can be made once and reused when necessary.

We propose a general algorithm for all possible values of $a$, and one for the special case $m=2a$. This is known in literature as Equally Spaced Polynomials. For comparison reasons we introduce instantiated algorithms for $a = 1$ and $a = m-1$, which are usually treated as special cases in the literature.

\subsection{Case $(m-a) \mid a$}

Situations where $m-a$ divides $a$ result in cancelled items, coefficients that are XORed twice to the same bit. The most visible case is when $m-a = a$, the equally-spaced trinomial. In this case, the cancelled items translate directly to cancelled operations, and the reduction can be performed in less steps.

Other cases can be found by using the formula $a = \frac{\alpha}{\alpha + 1}m$ with $a, \alpha \in \mathbb{Z}^{+}$ [I think it should be $\mathbb{N}$, Custodio disagrees]. The equally spaced case happens when $\alpha = 1$. As an example of other cases, $x^{20}+x^{15}+1$, $x^{20}+x^{16}+1$ and $x^{20}+x^{18}+1$ all contain some cancellations. The number of XOR cancellations is $m - a - 1$, which explains the complexity of $\frac{3}{2}m - 3$ for the equally-spaced case. It's still an open problem if the non-equally spaced case can be optimized at all.

\subsection{Case $a=1$}
When $a=1$, the number of reduction steps given by the Eq.~\ref{eq:k} is $$k=\left \lfloor \frac{m-2}{m-a} \right \rfloor + 1=\left \lfloor \frac{m-2}{m-1} \right \rfloor + 1=1.$$ This is the simplest case. The number of XOR operations is $N_\oplus = 2m-2$.

 \begin{algorithm}
 \begin{algorithmic}[1]
  \REQUIRE $m$, $C[2m-2,0]$
  \ENSURE $C[m-1,0]$
  \FOR{$i \gets m-2$ \textbf{downto} $0$}
      \STATE $C[i+1] \gets C[i+1] \oplus C[i+m]$
      \STATE $C[i] \gets C[i] \oplus C[i+m]$
  \ENDFOR
  \RETURN $C$
  \caption{Simple reduction algorithm for $x^m + x + 1$, $a = 1$}
  \label{alg:a1:bit:operation}
\end{algorithmic}
\end{algorithm}

Algorithm \ref{alg:a1:bit:operation}  uses $2m - 2$ operations and has a depth of only 2. This is the best possible case for a bit-parallel circuit.

\subsection{Case $ 1 < a < \frac{m}{2}$}
The number of reduction steps given by the Eq.~\ref{eq:k} is $$k=\left \lfloor \frac{m-2}{m-a} \right \rfloor + 1=2.$$ The number of XOR operations is $N_\oplus = 2m-2$.

\subsection{Case $m=2a$}
The special case where $m=2a$ operation algorithm is presented in Algorithm~\ref{alg:esp}. This algorithm, known in literature as Equally Spaced Trinomial, is essentially the same stated by Wu\cite[p. 753, Eq. 3]{Wu2002}. There, however, it used different vectors: one for the element to be reduced and other for the result itself. The number of reduction steps given by the Eq.~\ref{eq:k} is $$k=\left \lfloor \frac{m-2}{m-a} \right \rfloor + 1=\left \lfloor \frac{2a-2}{2a-a} \right \rfloor + 1=\left \lfloor \frac{2a-2}{a} \right \rfloor + 1=2.$$ 

\ref{alg:equallyspaced:bit:operation} is a $m=2a$ version of algorithm \ref{alg:general:bit:operation} with cancellations. It is trivial to see that its complexity is $m + a - 1$, or, since $a = m/2$, the number of operations is exactly $\frac{3}{2}m - 1$.

\begin{algorithm}
\begin{algorithmic}[1]
  \REQUIRE $a$, $C[0,2m-2]$
  \ENSURE $C[m-1,0]$
  \FOR{$i \gets 0$ \textbf{to} $m-2$}
    \STATE $C[i+a] \gets C[i+a] \oplus C[i+m]$
    \ENDFOR
  \FOR{$i \gets a - 1$ \textbf{to} $0$}
    \STATE $C[i] \gets C[i] \oplus C[i+m]$
    \ENDFOR
  \RETURN $C$
  \caption{Simple Reduction algorithm for $x^{m} + x^a +1$, $a = \frac{m}{2}$.}
  \label{alg:equallyspaced:bit:operation}
\end{algorithmic}
\end{algorithm}

\subsubsection{Proof}

Applying the equation of maximum number of reduction steps, by Sunar and Ko\c{c} \cite{SunarKoc}, to the equally spaced case we arrive at

\begin{equation}
  k = \left \lfloor \frac{m-2}{m-\frac{m}{2}} \right \rfloor + 1 = 2
\end{equation}

Therefore the reduction equation is comprised of two steps. Since $(x^m)^n = (x^a + 1)^n$, the full reduction formula for the equally spaced trinomial is:

\begin{gather}
    \sum_{i = 0}^{m-1}{c_i x^i} +
        \sum_{j = 0}^{\frac{m}{2} - 2}{c_{j + m} x^j} +
        \sum_{j = 0}^{m - 2}{c_{j + m} x^{j+\frac{m}{2}}} +
        \sum_{j = 0}^{\frac{m}{2} - 2}{c_{j + m + a} x^j}
\end{gather}

Lines 1 and 2 of algorithm \ref{alg:equallyspaced:bit:operation} correspond to the first reduction step for $x^a$ (last summation). Lines 3 and 4 correspond truncated first step for $x^0$ (second summation). And because of the mutable nature of the data structure, lines 3 and 4 operate on bits that were already XOR'ed in the first loop. This overlap gives raises to the remaining $a$ elements of the second reduction step for $x^0$ (third summation).

End of proof.

\begin{algorithm}
\begin{algorithmic}[1]
  \REQUIRE $a$, $C[4a-2,0]$
  \ENSURE $C[2a-1,0]$
  \FOR{$i \gets 0$ \textbf{to} $a-2$}
    \STATE $C[i] \gets C[i] \oplus C[i+2a] \oplus C[i+3a]$
    \ENDFOR
  \STATE $C[a-1] \gets C[a-1] \oplus C[3a-1]$ 
  \FOR{$i \gets a$ \textbf{to} $2a-1$}
    \STATE $C[i] \gets C[i] \oplus C[i+a]$
  \ENDFOR
  \RETURN $C$
  \caption{Reduction algorithm for $x^{m} + x^a +1$, $m=2a$.}
  \label{alg:esp}
\end{algorithmic}
\end{algorithm}

\subsection{Case $a=m-1$}
The special case where $a=m-1$ operation algorithm is presented in Algorithm~\ref{alg:ma1}. This algorithm has the same complexity as the general algorithm, i.e, $N_\oplus = 2m-2$. The number of reduction steps given by the Eq.~\ref{eq:k} is $$k=\left \lfloor \frac{m-2}{m-a} \right \rfloor + 1 = \left \lfloor \frac{m-2}{m-(m-1)} \right \rfloor + 1 = \left \lfloor m-2 \right \rfloor + 1=m-1.$$

\begin{algorithm}
\begin{algorithmic}[1]
  \REQUIRE $a$, $C[2m-2,0]$
  \ENSURE $C[m-1,0]$
  \FOR{$i \gets 2m-3$ \textbf{downto} $m$}
    \STATE $C[i] \gets C[i] \oplus C[i+1]$
    \ENDFOR
  \STATE $C[a] \gets C[a] \oplus C[m]$ 
  \FOR{$i \gets 0$ \textbf{to} $m-2$}
    \STATE $C[i] \gets C[i] \oplus C[i+m]$
  \ENDFOR
  \RETURN $C$
  \caption{Reduction algorithm for $x^{m} + x^a +1$, $a=m-1$.}
  \label{alg:ma1}
\end{algorithmic}
\end{algorithm}

\subsection{First General case for $1 \leq a < m$}
The general bit operation algorithm is presented in Algorithm~\ref{alg:general:bit:operation}.

\begin{algorithm}
\begin{algorithmic}[1]
  \REQUIRE $m$, $a$, $C[2m-2,0]$
  \ENSURE $C[m-1,0]$
  \FOR{$i \gets m-2$ \textbf{downto} $0$}
      \STATE $C[i+a] \gets C[i+a] \oplus C[i+m]$
      \STATE $C[i] \gets C[i] \oplus C[i+m]$
  \ENDFOR
  \RETURN $C$
  \caption{Simple reduction algorithm for $x^m + x^a +1$, $m \not = 2a$}
  \label{alg:general:bit:operation}
\end{algorithmic}
\end{algorithm}

This algorithm obviously uses $2 (m - 1) = 2m -2$ XORs.

\section{Word operation reduction algorithms}

\subsection{Notations}
In this paper, we assume that one word has $W$ bits where $W$ is a multiple of $8$. The bits of a word $A$ are numbered from $0$ to $W-1$, with the rightmost bit (LSB) of $A$ designated as bit $0$. The following standard notation is used to denote operations on words $A$ and $B$:\\

\begin{tabular}{ll}
  $A \BitXOR B$ & bitwise exclusive or. \\
  $A \BitOr B$ & bitwise or. \\
  $A \BitAnd B$     & bitwise $AND$. \\ 
  $A \ShiftLeft n$     & left shift of $A$ by $n$ positions, ($n<W$), with the bits from 0 to $n-1$ set to $0$. \\
  $A\ShiftRight n$    & righ shift of $A$ by $n$ positions, ($n<W$), with bits from $W-n$ to $W-1$ set to $0$. \\
  $C$ & Vector os bits.\\
  $D$ & Array de elements of $W$ bits. \\
  $D_{Red}$ & Reduced element. \\
\end{tabular}\\

Figure~\ref{fig:elemento:field} shows the representation of a element $d \in \ftwom$ as an array $D$ of $t$ words of $W$ bits, where $t = \left \lceil \frac{m}{W} \right \rceil$. The $s = tW-m$ highest order bits of $D[t-1]$ are not unused.
\begin{figure}[htb]
  \centering
  \includegraphics[width = .55\columnwidth]{figures/element-word.pdf}
\caption{Representation of $d \in \ftwom$ as an array $D$ of $W$ bit words. The $s = tW-m$ highest order bits of $D[t-1]$ are not unused.}
\label{fig:elemento:field}
\end{figure}
\\

For the case where $(m-1) \mod{W} > \frac{W}{2}$, the Figure~\ref{fig:elemento:field:mult} shows the representation of a element $d = ab$, $a,b \in \ftwom$ as an array $D$ of $W$-bit words. The $s = 2(tW-m)+1$ highest order bits of $D[2t-1]$ are not unused.

\begin{figure}[htb]
  \centering
  \includegraphics[width = .9\columnwidth]{figures/two-word-element-1.pdf}
\caption{Representation of $d = ab$, $a,b \in \ftwom$ as an array $D$ of $W$-bit words. The $s = 2(tW-m)+1$ highest order bits of $D[2t-1]$ are not unused.}
\label{fig:elemento:field:mult}
\end{figure}


When $(m-1) \mod{W} \leq \frac{W}{2}$, $D[2t-1]$ is not needed.  Figure~\ref{fig:elemento:field:mult2} shows the representation of a element $d = ab$, $a,b \in \ftwom$ as an array $D$ of $W$-bit words. The $s = 2(tW-m)+1$ highest order bits of $D[2t-2]$ and $D[2t-1]$ are not unused.
\begin{figure}[htb]
  \centering
  \includegraphics[width = \columnwidth]{figures/two-word-element-2.pdf}
\caption{Representation of $d = ab$, $a,b \in \ftwom$ as an array $D$ of $W$-bit words. The $s = 2(tW-m)+1$ highest order bits of $D[2t-2]$ and $D[2t-1]$ are not unused.}
\label{fig:elemento:field:mult2}
\end{figure}

In these cases a whole word is wasted, but length calculations become simpler: to get the length (in words) of a product of two elements, just add their lengths (in words).

\subsection{Algorithms}
The processing algorithms by word found in the literature for the irreducible polynomials are an implementation using words of the general algorithm~\ref{alg:bits} of bits. These algorithms, therefore, do not take into account the possible redundant operations inside. One explanation for this could be the fact that for the small middle exponents, this redundancy is small, almost negligible. However, for large exponents the redundancy increases considerably. If the algorithm benefit from these redundancies, you can get algorithms as effective as, or perhaps even better, for irreducibles with great exponents. \\

{\bf Custodio} Teremos nessa seção: \\
a) Reducao por bit usando palavras\\
b) Redução usando palavras do algoritmo geral de bits;\\
c) Algoritmos do NITS por palavra;\\
d) Nossos algoritmos otimizados ( $a=m/2$, $a \neq m/2$ ) processados por palavras;\\
e) Novos algoritmos para os trinomios do NITS mas usando feito usando nosso algoritmo otimizado para a primeira faixa de $a$;\\
 

\subsection{General by word bit processing}
For low weight irreducible polynomials with middle terms close to each other or, in particular, trinomials, the reduction may be performed efficiently by words as shown in Algorithm~\ref{alg:wordbit:hankerson}\cite[p. 53]{Hankerson}. In this algorithm $$r(x) = f(x) + x^m$$\\

{\bf Custodio} Verificar o algorithm. Explicar ( modificar ) $C\{j\}$. Fazer uma figura semelhante a Fig. 2.9 do Hankerson para explicar de forma geral este algoritmo. Apresentar uma análise de complexidade para este algoritmo. Por exemplo, a soma re $u_k$ pode ser feita por um $for$ de $W$ passos.

\begin{algorithm}
\begin{algorithmic}[1]
  \REQUIRE $C[2m-2,0]$, $W$
  \ENSURE $C[m-1,0]$
  {\it Precomputation}. Compute $u_k = x^k r(x)$, $0 \leq k \leq W-1$ 
  \FOR{$i \gets 2m-2$ \textbf{downto} $m$}
    \IF{$C[i]=1$}
      \STATE $j \gets \left \lfloor \frac{i-m}{W} \right \rfloor$
      \STATE $k \gets (i-m) - Wj$
      Add $u_k(x)$ to $C\{j\}$
    \ENDIF
  \ENDFOR
  \RETURN $C$
  \caption{Reduction algorithm by word (one bit at a time) (Hankerson).}
  \label{alg:wordbit:hankerson}
\end{algorithmic}
\end{algorithm}

\subsection{Word operation of the case $a=1$}\label{sec:word:operation:alg:a1}
Figure \ref{fig:word:operation:alg:a1} is a representation of the case $a=1$ and $(m-1) \mod{W} > \frac{W}{2}$. In the figure, $D$ is the element to be reduced, $D_a$ are the bits reduced by the exponent $a$ and $D_0$ are the bits reduced by $0$. $D_{Red} = D \BitXOR D_a \BitXOR D_0$ is the reduced element. The hatched bits (shaded in figure) should be set to 0.
\begin{figure}[htb]
  \centering
  \includegraphics[width = \columnwidth]{figures/two-elements-reduction-a-1.pdf}
\caption{Word operation of the case $a=1$.}
\label{fig:word:operation:alg:a1}
\end{figure}
\\

\begin{algorithm}
  \begin{algorithmic}[1]
  \REQUIRE $D[2t-1,0]$,$t$, $m$, $W$
  \ENSURE $D_{Red = D[t-1,0]}$
  \STATE $u \gets tW - m$
  \STATE $h \gets W - u$
  \STATE $T \gets D[t-1] \ShiftRight h$
  \STATE $D[0] \gets D[0] \oplus T \oplus (D[t] \ShiftLeft u) \oplus (T \ShiftLeft 1) \oplus (D[t] \ShiftLeft (u+1)$ \label{alg:a1:high:primeira-linha}
  \FOR{$i \gets 1$ \textbf{to} $t-2$}
    \STATE $D[i] \gets D[i] \oplus (D[i+t-1] \ShiftRight h) \oplus (D[i+t-1] \ShiftRight (h-1)) \oplus (D[i+t] \ShiftLeft u) \oplus (D[i+t] \ShiftLeft (u+1))$
    \ENDFOR
  \STATE $T \gets D[2t-1] \ShiftLeft (2u+1)$\ \COMMENT{This avoid using \& to reset unused bits}
  \STATE $D[t-1] \gets D[t-1] \oplus (D[2t-2] \ShiftRight h) \oplus (D[2t-2] \ShiftRight (h-1)) \oplus (T \ShiftRight u) \oplus (T \ShiftRight (u+1))$
  \RETURN $D_{Red}$
  \caption{Reduction algorithm by word for $x^m + x + 1$, from $D[2t-1]$.}
  \label{alg:word:a1:high}
\end{algorithmic}
\end{algorithm}

We have $4$ XORs in line \ref{alg:a1:high:primeira-linha}, $4(t-2)$ XORs in line 5, and $4$ XORs in line 8. The total is $$N_\oplus = 4t.$$

We have $$N_{Shifts} = 4t.$$

Now, we have the case where $(m-1) \mod{W} \leq \frac{W}{2}$. In this case, $D[2t-2]$ is not used as show in Figure \ref{fig:word:operation:alg:a1:low}. Thus, the algorithm procede to reduce the bits from $D[2t-1]$. The Algorithm is show in \ref{alg:word:a1:low}.

\begin{figure}[htb]
  \centering
  \includegraphics[width = \columnwidth]{figures/two-elements-reduction-a-1-low.pdf}
\caption{Word operation of the case $a=1$ where $D[2t-1]$ is unused.}
\label{fig:word:operation:alg:a1:low}
\end{figure}

\begin{figure}[htb]
  \centering
  \includegraphics[width = \columnwidth]{figures/reduction-equally-spaced.pdf}
\caption{Word operation of the Algorithm \ref{alg:equallyspaced:word:operation} where $D[2t-1]$ is used.}
\label{fig:word:operation:alg:a1:low2}
\end{figure}

\begin{algorithm}
  \begin{algorithmic}[1]
  \REQUIRE $D[2t-2,0]$,$t$, $m$, $W$
  \ENSURE $D_{Red = D[t-1,0]}$
  \STATE $u \gets tW - m$
  \STATE $h \gets W - u$
  \STATE $T \gets D[t-1] \ShiftRight h$
  \STATE $D[0] \gets D[0] \oplus T \oplus (D[t] \ShiftLeft u) \oplus (T \ShiftLeft 1) \oplus (D[t] \ShiftLeft (u+1)$ \label{alg:a1:low:primeira-linha}
  \FOR{$i \gets 1$ \textbf{to} $t-3$}
    \STATE $D[i] \gets D[i] \oplus (D[i+t-1] \ShiftRight h) \oplus (D[i+t-1] \ShiftRight (h-1)) \oplus (D[i+t] \ShiftLeft u) \oplus (D[i+t] \ShiftLeft (u+1))$
    \ENDFOR
  \STATE $T \gets D[2t-2] \ShiftLeft (2u+1-W)$\ \COMMENT{This avoid using \& to reset unused bits}
  \STATE $D[t-2] \gets D[t-2] \oplus (D[2t-3] \ShiftRight h) \oplus (D[2t-3] \ShiftRight (h-1)) \oplus (T \ShiftRight u) \oplus (T \ShiftRight (u+1))$
  \RETURN $D_{Red}$
  \caption{Reduction algorithm by word for $x^m + x + 1$, from $D[2t-2]$.}
  \label{alg:word:a1:low}
\end{algorithmic}
\end{algorithm}

We have $4$ XORs in line \ref{alg:a1:low:primeira-linha}, $4(t-3)$ XORs in line 5, and $4$ XORs in line 8. The total is $$N_\oplus = 4t-4.$$

We have $$N_{Shifts} = 4t-4.$$

\subsection{Word operation of the case $a<m/2$}\label{sec:word:operation:alg:a2}
Figure \ref{fig:word:operation:alg:a2} is a representation of the case $a<m/2$. In the figure, $D$ is the element to be reduced, $D_a$ are the bits reduced by the exponent $a$ and $D_0$ are the bits reduced by $0$. $D_{Red} = D \BitXOR D_a \BitXOR D_0$ is the reduced element. The hatched bits (shaded in figure) should be set to 0.
\begin{figure}[htb]
  \centering
  \includegraphics[width = \columnwidth]{figures/two-elements-reduction-a-6.pdf}
\caption{Word operation of the case $a<m/2$.}
\label{fig:word:operation:alg:a2}
\end{figure}
\\

\subsection{Word operation of the  Algorithm~\ref{alg:esp}}\label{sec:word:operation:alg:esp}
Figure \ref{fig:word:operation:alg:esp} is a representation of the algorithm \ref{alg:esp}. In the figure, $D$ is the element to be reduced, $D_a$ are the bits reduced by the exponent $a$ and $D_0$ are the bits reduced by $0$. $D_{Red} = D \BitXOR D_a \BitXOR D_0$ is the reduced element. The hatched bits (shaded in figure) should be set to 0.
\begin{figure}[htb]
  \centering
  \includegraphics[width = \columnwidth]{figures/two-elements-reduction-a-11.pdf}
\caption{Word operation of the Algorithm \ref{alg:esp}.}
\label{fig:word:operation:alg:esp}
\end{figure}
\\

\subsection{Word operation of the  Algorithm~\ref{alg:general:bit:operation}}\label{sec:word:operation}

The word-oriented versions of the algorithm can be created by noticing the bit access pattern. The XOR operation pattern can be thought of XORing two ranges

Each of them perform a XOR operation between all bits in a certain range and counterparts a fixed distance away. This access pattern allows word-level parallelism. There are four issues that need to be paid attention to:

\textbf{Misaligned start:}
The loop starts at $2m -2$, which may not align with the word length ($W \nmid 2m-2$). To avoid overwriting the upper bits of this incomplete word we must clear them from the \texttt{src} word.

\textbf{Misaligned ends:}
The fix is identical to the misaligned start, where we clear the extra bits from the incoming word.

\textbf{Misaligned source:}
The source/incoming word may not be aligned to word boundaries either. In this case we need to pull bits from the next word too, shifting both of them for correct placement. Unfortunately this issue affects the start and ends word too, further complicating their treatment.

\textbf{Too small distances:}
Finally, if the distance between the XOR'ed words ($m-a$, $m$, respectively) is smaller than $W/2$, the same bit will be written and read in the same word-step. This is not possisble with a single operation, and requires handling each word in mulitple parts. This is suboptimal and should be avoided by better choices of $m$, $a$ and $W$.

Other bit-level operations are emulated with word masks and shifts.

To help transforming an algorithm in its word-parallel version we introduce the following helper functions. Notice the function calls can be inlined and all conditionals are open for static analysis (the final algorithm should contain no conditionals if optimized properly).

\begin{algorithm}
\begin{algorithmic}[1]
  \REQUIRE $dst$, $src$, $W$, $C[]$
  \STATE $mask \gets 1 << (src \mod W)$
  \STATE $C[dst / W] \gets C[dst / W] \oplus (C[src / W] \land mask) << (dst \mod W - src \mod W)$
  \caption{\texttt{XOR\_BIT}: Single bit XOR inside word}
\end{algorithmic}
\end{algorithm}

\begin{algorithm}
\begin{algorithmic}[1]
  \REQUIRE $dst$, $src$, $W$, $C[]$
  \STATE $left \gets C[src / W] >> (src \mod W)$
  \STATE $right \gets C[src / W + 1] << (W - src \mod W)$
  \STATE $C[dst / W] \gets C[dst / W] \oplus (left \lor right)$
  \caption{\texttt{XOR\_WORD}: Whole word XOR with possibly misaligned source}
\end{algorithmic}
\end{algorithm}

\begin{algorithm}
\begin{algorithmic}[1]
  \REQUIRE $dst$, $src$, $n\_bits$, $W$, $C[]$
  \STATE $shift \gets src \mod W$
  \STATE $left\_n\_bits \gets \min(n\_bits, W - shift)$
  \STATE $leftmask \gets ((1 << left\_n\_bits) - 1) << shift$
  \STATE $left \gets (C[src / W] \land leftmask) >> shift$
  \STATE $right\_n\_bits \gets n\_bits - left\_n\_bits$
  \IF{$right\_n\_bits > 0$}
    \STATE $rightmask \gets (1 << right\_n\_bits) - 1$
    \STATE $right \gets (C[src/W+1] \land rightmask) << left\_n\_bits$
  \ELSE
    \STATE $right \gets 0$
  \ENDIF
  \STATE $C[dst/W] = C[dst/W] \oplus ((left | right) << (dst \ W))$
  \caption{\texttt{XOR\_PARTIAL\_WORD}: XOR a range of bits inside a word, with possibly misaligned source}
\end{algorithmic}
\end{algorithm}

\begin{algorithm}
\begin{algorithmic}[1]
  \REQUIRE $start\_dst$, $end\_dst$, $distance$, $W$, $C[]$
  \IF{$end\_dst/W = start\_dst/W$}
    \STATE \texttt{XOR\_PARTIAL\_WORD}($start\_dst$, $start\_dst + distance$, $end\_dst - start\_dst$, $W$, $C$)
  \ELSE
    \IF{$start\_dst \ W \neq 0$}
      \STATE $remaining = W - (start\_dst \ W)$
      \STATE \texttt{XOR\_PARTIAL\_WORD}($start\_dst$, $start\_dst + distance$, $remaining$, $W$, $C$)
      \STATE $start\_dst \gets start\_dst + remaining$
    \ENDIF
    
    \STATE $rounded\_end \gets (end\_dst / W) * W$
    \STATE $dst \gets start\_dst$
    \WHILE{$dst < rounded\_end$}
      \STATE \texttt{XOR\_WORD}($dst$, $dst + distance$, $W$, $C$)
      \STATE $dst \gets dst + W$
    \ENDWHILE
    
    \IF{$end\_dst \ W \neq 0$}
      \STATE \texttt{XOR\_PARTIAL\_WORD}($rounded\_end$, $rounded\_end + distance$, $end\_dst \ W$, $W$, $C$)
    \ENDIF
  \ENDIF
  \caption{\texttt{XOR\_RANGE}: XOR a range of bits (across many words) with an equivalent range a certain distance away}
\end{algorithmic}
\end{algorithm}


And the word-transformed algorithms are:

\begin{algorithm}
\begin{algorithmic}[1]
  \REQUIRE $m$, $a$, $W$, $C\left[0,\ceil*{\frac{2m-2}{W}}\right]$
  \ENSURE $C[0,m-1]$
  \STATE \texttt{XOR\_RANGE}($a$, $a+m$, $m$, $W$, $C$)
  \STATE \texttt{XOR\_RANGE}($0$, $a-1$, $m$, $W$, $C$)
  \RETURN $C$
  \caption{Simple word-parallel reduction algorithm for $x^m + x^a +1$, $a = \frac{m}{2}$}
  \label{alg:equallyspaced:word:operation}
\end{algorithmic}
\end{algorithm}

\begin{algorithm}
\begin{algorithmic}[1]
  \REQUIRE $m$, $a$, $W$, $C\left[0,\ceil*{\frac{2m-2}{W}}\right]$
  \ENSURE $C[0,m-1]$
  \STATE $start\_src \gets m$
  \STATE $step\_size \gets m - a$
  \WHILE{$start\_src \le 2m - 2$}
    \STATE \texttt{XOR\_RANGE}($a$, $a + 2m-2 - start\_src$, $start\_src - a$, $W$, $C$)
    \STATE $start\_src \gets start\_src + step\_size$
    \STATE $step\_size \gets 2 \times step\_size$
  \ENDWHILE
  \STATE \texttt{XOR\_RANGE}($0$, $m$, $m$, $W$, $C$)
  \RETURN $C$
  \caption{Alternative reduction algorithm, uses more XORs but less calls to \texttt{XOR\_RANGE}, may be useful for certain architectures.}
  \label{alg:new:word:operation}
\end{algorithmic}
\end{algorithm}

\section{Example Algorithms}

The algorithm \ref{alg:new:word:operation} proposed can be applied to specific trinomials with fixed parameters. For example, for the commonly used field $x^{233} + x^{74} + 1$, with 32-bit words, this results in algorithm \ref{alg:233-32:word:operation}. The total number of operations of this algorithm is 19 XORs, 16 ORs, 5 ANDs and 35 shifts, for a total of 75 word operations.

\begin{algorithm}
\begin{algorithmic}[1]
  \REQUIRE $C\left[0,\ceil*{\frac{2m-2}{W}}\right]$
  \ENSURE $C[0,m-1]$
  \STATE $C[2] \gets C[2] \oplus ((C[7] \land \texttt{0x7ffffe00}) << 1)$
  \STATE $C[3] \gets C[3] \oplus ((C[7] >> 31) \lor (C[8] << 1))$
  \STATE $C[4] \gets C[4] \oplus ((C[8] >> 31) \lor (C[9] << 1))$
  \STATE $C[5] \gets C[5] \oplus ((C[9] >> 31) \lor (C[10] << 1))$
  \STATE $C[6] \gets C[6] \oplus ((C[10] >> 31) \lor (C[11] << 1))$
  \STATE $C[7] \gets C[7] \oplus ((C[11] >> 31) \lor (C[12] << 1))$
  \STATE $C[8] \gets C[8] \oplus ((C[12] >> 31) \lor (C[13] << 1))$
  \STATE $C[9] \gets C[9] \oplus ((C[13] >> 31) \lor ((C[14] \land \texttt{0x1ffff}) >> 1))$
  \STATE $C[2] \gets C[2] \oplus ((C[12] \land \texttt{0x3fffff00}) << 2)$
  \STATE $C[3] \gets C[3] \oplus ((C[12] >> 30) \lor (C[13] << 2))$
  \STATE $C[4] \gets C[4] \oplus ((C[13] >> 30) \lor ((C[14] \land \texttt{0x1ffff}) >> 2))$
  \STATE $C[0] \gets C[0] \oplus ((C[7] >> 9) \lor (C[8] << 23))$
  \STATE $C[1] \gets C[1] \oplus ((C[8] >> 9) \lor (C[9] << 23))$
  \STATE $C[2] \gets C[2] \oplus ((C[9] >> 9) \lor (C[10] << 23))$
  \STATE $C[3] \gets C[3] \oplus ((C[10] >> 9) \lor (C[11] << 23))$
  \STATE $C[4] \gets C[4] \oplus ((C[11] >> 9) \lor (C[12] << 23))$
  \STATE $C[5] \gets C[5] \oplus ((C[12] >> 9) \lor (C[13] << 23))$
  \STATE $C[6] \gets C[6] \oplus ((C[13] >> 9) \lor (C[14] << 23))$
  \STATE $C[7] \gets C[7] \oplus ((C[14] \land \texttt{0x3fe00}) >> 9)$
  \RETURN $C$
  \caption{Reduction algorithm for $x^{233} + x^{74} + 1$ with 32-bit words}
  \label{alg:233-32:word:operation}
\end{algorithmic}
\end{algorithm}


% http://download.springer.com/static/pdf/429/chp253A10.1007252F978-3-642-33481-8_10.pdf?originUrl=http3A2F2Flink.springer.com2Fchapter2F10.10072F978-3-642-33481-8_10&token2=exp=1441741030~acl=2Fstatic2Fpdf2F4292Fchp25253A10.100725252F978-3-642-33481-8_10.pdf3ForiginUrl3Dhttp253A252F252Flink.springer.com252Fchapter252F10.1007252F978-3-642-33481-8_10*~hmac=7fed14b6b9363e3a95601342a355c7f5b9a630f6dcad462d10adb7e8ff1552b9
%http://ieeexplore.ieee.org/xpls/abs_all.jsp?arnumber=6927388&tag=1
%http://www.sciencedirect.com/science/article/pii/S107157971400121X
%https://eprint.iacr.org/2007/192.pdf


\section{NIST Polynomials and their costs}

NIST Binary Polynomials and their costs (Guide to Elliptic Curve Crytography - Hankerson - page 76

\begin{itemize}
\item $x^{163} + x^7 + x^6 + x^3 + 1$ - $C[0..10]$, 41 XORs, 36 SHIFTs, 1 AND = 78 operations
\item $x^{233} + x^{74} + 1$ - $C[0..15!]$, 35 XORs, 35 SHIFTs, 1 AND = 71 operations
\item $x^{283} + x^{12} + x^7 + x^5 + 1$ - $C[0..17]$, 76 XORs, 76 SHIFTs, 1 AND = 153 operations
\item $x^{409} + x^{87} + 1$ - $C[0..25]$, 54 XORs, 54 SHIFTs, 1 AND = 109 operations
\item $x^{571} + x^{10} + x^5 + x^2 + 1$ - $C[0..35]$, 148 XORs, 148 SHIFTs, 1 AND = 297 operations
\end{itemize}


Pentanomials of same degree from the family $x^{b+2c} + x^{b+c} + x^b + x^c + 1$

\begin{itemize}
\item $x^{233} + x^{158} + x^{83} + x^{74} + 1$, score 3.717, $k_a=4$
\item $x^{163} + x^{117} + x^{71} + x^{46} + 1$, score 3.982, $k_a=4$
\item $x^{409} + x^{294} + x^{179} + x^{115} + 1$, score 4.015, $k_a=4$
\item $x^{409} + x^{337} + x^{265} + x^{72} + 1$, score 4.457, $k_a=6$
\item $x^{233} + x^{217} + x^{201} + x^{16} + 1$, score 7.240, $k_a=15$
\item ??571??
\end{itemize}

% "joguinho" boppreh.com
\chapter{Visual debugger}
\begin{figure}[htb]
  \centering
  \includegraphics[width=0.8\columnwidth]{figures/reduction_matrix.png}
\caption{Reduction matrix for $x^{13}+x^8+1$, the first step of the XOR Count algorithm. By replacing the numbers on the matrix with the input bits, the reduced polynomial appears on the last line with XORs computed column-wise.}
\label{fig:reduction_matrix}
\end{figure}


% Discutir as contribuições. O que foi trazido de melhorias? Avaliação do trabalho. Contextualizar a contribuição, com análise crítical.
\chapter{Evalution?}
\input{chapters/evaluation}

% Objetivos atingidos: específicos e gerais.
% Propostas futuras.
\chapter{Final considerations}
% Relacionar os objetivos específicos ( cada um deles ) com o que foi feito e onde está isso no trabalho, de forma critica
% Trabalhos futuros: pelos menos apresentar 5 trabalhos futuros

\section{Conclusion} \label{conclusion}

In this work we present a low-complexity bit-parallel squarer algorithm. It is able to match the state of the art~\cite{wu2002bit} in XOR gate count when $GF(2^m)$ is defined using trinomials. Xiong~\cite{xiong2014gf} achieves lower XOR gate complexity when pentanomials are used; however it uses a special polynomial basis and is suitable to only a small fraction of all irreducible pentanomials, while ours is general for any low-weight polynomial.

Park~\cite{park2012explicit} also proposes a general squaring algorithm focusing on pentanomials with $a \leq \ceil{m/2}$. Her squarer is the best available in the literature considering both the number of XOR operations and the delay. However, our algorithm is able to match the circuit delay on some polynomials and has a strictly smaller number of XORs than the upper bound provided, while being suitable not only for pentanomials. 
We remark that our squarer works for any general low-weight $k$-nomial. As mentioned in Section~\ref{squaring} there is not much work in the literature for this type of $k$-nomial squarer, $k>5$, but our algorithm can still be efficiently used in these cases.

One disadvantage of our proposed squarer is that it may produce higher delays, up to $6 T_X$ for pentanomials. This can be mitigated in two ways. First, by narrowing the selection of irreducible pentanomial. Indeed,  pentanomials that result in a delay of $3 T_X$ are abundant. Second, the resulting circuit can be modified to reduce the delay in exchange for some extra XORs. An automated way to perform this tweak is still an open problem, as is the exact formula for circuit delays.

In conclusion, our algorithms give more freedom to select irreducible polynomials while maintaining competitive efficiency.





\section{References}

\bibliographystyle{ufsc-alf}
\bibliography{aaabibliografia}

\begin{appendices}

\section{Example of resulting algorithm}
\label{appendix:example}

\begin{algorithm}
\caption{Squaring for $\F_{2^{19}} \cong \F_2[x]/(x^{19} + x^5 + x^2 + x + 1)$}
\label{alg:square:example}
\begin{multicols}{2}
\begin{algorithmic}[1]
\REQUIRE $A = [a_0, a_1, a_2, ..., a_{18}]$, \\$f(x)~=~x^{19}~+~x^5~+~x^2~+~x~+~1$
\ENSURE $A^2 \mod f$
\STATE $C = [a_0, 0, a_1, 0, a_2, 0, \ldots, 0, a_{18}]$
\STATE $\mathbox{C[22]}{C[22]} \leftarrow C[22]  \oplus  C[36]$
\STATE $\mathbox{C[22]}{C[19]} \leftarrow C[36]$
\STATE $\mathbox{C[22]}{C[18]} \leftarrow C[18]  \oplus  C[36]$
\STATE $\mathbox{C[22]}{C[17]} \leftarrow C[36]$
\STATE $\mathbox{C[22]}{C[20]} \leftarrow C[20]  \oplus  C[34]$
\STATE $\mathbox{C[22]}{C[17]} \leftarrow C[17]  \oplus  C[34]$
\STATE $\mathbox{C[22]}{C[16]} \leftarrow C[16]  \oplus  C[34]$
\STATE $\mathbox{C[22]}{C[15]} \leftarrow C[34]$
\STATE $\mathbox{C[22]}{C[18]} \leftarrow C[18]  \oplus  C[32]$
\STATE $\mathbox{C[22]}{C[15]} \leftarrow C[15]  \oplus  C[32]$
\STATE $\mathbox{C[22]}{C[14]} \leftarrow C[14]  \oplus  C[32]$
\STATE $\mathbox{C[22]}{C[13]} \leftarrow C[32]$
\STATE $\mathbox{C[22]}{C[16]} \leftarrow C[16]  \oplus  C[30]$
\STATE $\mathbox{C[22]}{C[13]} \leftarrow C[13]  \oplus  C[30]$
\STATE $\mathbox{C[22]}{C[12]} \leftarrow C[12]  \oplus  C[30]$
\STATE $\mathbox{C[22]}{C[11]} \leftarrow C[30]$
\STATE $\mathbox{C[22]}{C[14]} \leftarrow C[14]  \oplus  C[28]$
\STATE $\mathbox{C[22]}{C[11]} \leftarrow C[11]  \oplus  C[28]$
\STATE $\mathbox{C[22]}{C[10]} \leftarrow C[10]  \oplus  C[28]$
\STATE $\mathbox{C[22]}{C[9]} \leftarrow C[28]$
\STATE $\mathbox{C[22]}{C[12]} \leftarrow C[12]  \oplus  C[26]$
\STATE $\mathbox{C[22]}{C[9]} \leftarrow C[9]  \oplus  C[26]$
\STATE $\mathbox{C[22]}{C[8]} \leftarrow C[8]  \oplus  C[26]$
\STATE $\mathbox{C[22]}{C[7]} \leftarrow C[26]$
\STATE $\mathbox{C[22]}{C[10]} \leftarrow C[10]  \oplus  C[24]$
\STATE $\mathbox{C[22]}{C[7]} \leftarrow C[7]  \oplus  C[24]$
\STATE $\mathbox{C[22]}{C[6]} \leftarrow C[6]  \oplus  C[24]$
\STATE $\mathbox{C[22]}{C[5]} \leftarrow C[24]$
\STATE $\mathbox{C[22]}{C[8]} \leftarrow C[8]  \oplus  C[22]$
\STATE $\mathbox{C[22]}{C[5]} \leftarrow C[5]  \oplus  C[22]$
\STATE $\mathbox{C[22]}{C[4]} \leftarrow C[4]  \oplus  C[22]$
\STATE $\mathbox{C[22]}{C[3]} \leftarrow C[22]$
\STATE $\mathbox{C[22]}{C[6]} \leftarrow C[6]  \oplus  C[20]$
\STATE $\mathbox{C[22]}{C[3]} \leftarrow C[3]  \oplus  C[20]$
\STATE $\mathbox{C[22]}{C[2]} \leftarrow C[2]  \oplus  C[20]$
\STATE $\mathbox{C[22]}{C[1]} \leftarrow C[20]$
\STATE $\mathbox{C[22]}{C[5]} \leftarrow C[5]  \oplus  C[19]$
\STATE $\mathbox{C[22]}{C[2]} \leftarrow C[2]  \oplus  C[19]$
\STATE $\mathbox{C[22]}{C[1]} \leftarrow C[1]  \oplus  C[19]$
\STATE $\mathbox{C[22]}{C[0]} \leftarrow C[0]  \oplus  C[19]$
\RETURN $C[0],~C[1],~C[2],~\ldots,~C[18]$
\end{algorithmic}
\end{multicols}
\end{algorithm}

Algorithm~\ref{alg:square:example} is an instance of Algorithm~\ref{alg:square}, wherein the irreducible polynomial is defined as $x^{19}+x^5+x^2+x+1$ by simply fixing $f$. We note the circuit has a delay of $3T_X$, due to the critical paths of the circuit lines $C[5]$ and $C[2]$. However, this value can be reduced with manual tweaks. By eliminating Lines 29, 31 and 38, and inserting $C[5] \leftarrow C[22] \oplus C[24]$ and $C[5] \leftarrow C[5] \oplus C[19]$ at the beginning, the delay of $C[5]$ is reduced to $2 T_X$ (and one XOR is saved). By moving Line 39 to before Line 36 the critical path of $C[2]$ is also reduced to only $2 T_X$, which becomes the total delay of the full circuit.

\end{appendices}

\end{document}
